%%%%%%%%%%%%%%%%%%%%%%%%%%%%%%%%%%%%%%%%%%%%%%%%%%%%%%%%%%%%%%%%%%%
%  File name: ch1-engl.tex
%  Title:
%  Version: 10.09.2015 (hve)
%%%%%%%%%%%%%%%%%%%%%%%%%%%%%%%%%%%%%%%%%%%%%%%%%%%%%%%%%%%%%%%%%%%
%%%%%%%%%%%%%%%%%%%%%%%%%%%%%%%%%%%%%%%%%%%%%%%%%%%%%%%%%%%%%%%%%%%
\chapter[Vector algebra in Euclidian space ${\mathbb R}^{n}$
]{Vector algebra in Euclidian space ${\mathbb R}^{n}$}
\lb{ch1}
%%%%%%%%%%%%%%%%%%%%%%%%%%%%%%%%%%%%%%%%%%%%%%%%%%%%%%%%%%%%%%%%%%%
Let us begin our elementary considerations of {\bf vector algebra} 
with the introduction of a special class of mathematical objects. 
These will be useful at a later stage, when we turn to formulate 
certain problems of a quantitative nature in a compact and elegant 
way. Besides introducing these mathematical objects, we also need 
to define which kinds of mathematical operations they can be 
subjected to, and what computational rules we have to take care of.

%%%%%%%%%%%%%%%%%%%%%%%%%%%%%%%%%%%%%%%%%%%%%%%%%%%%%%%%%%%%%%%%%%%
\section[Basic concepts]%
{Basic concepts}
\lb{sec:vekeinf}
%%%%%%%%%%%%%%%%%%%%%%%%%%%%%%%%%%%%%%%%%%%%%%%%%%%%%%%%%%%%%%%%%%%
Given be a set $V$ of mathematical objects $\vec{a}$ which, for now, we want to consider merely as a collection of $n$ arbitrary real numbers $a_{1}$, \ldots, $a_{i}$, \ldots, $a_{n}$. In explicit terms,
%
\be
\lb{vecsp}
V =  \left\{
\vec{a} = \left(
\begin{array}{c}
a_{1} \\ \vdots \\ a_{i} \\ \vdots \\ a_{n}
\end{array}\right)
\left|\, a_{i} \in {\mathbb R},\hspace{1mm}
i=1, \dots, n\right.\right\} \ .
\ee
%
Formally the $n$ real numbers considered can either be assembled in an ordered pattern as a column or a row. We define

\medskip
\noindent
\underline{\bf Def.:}
Real-valued {\bf column vector} with $n$ 
components
%
\be
\fbox{$\displaystyle
\vec{a} := \left(
\begin{array}{c}
a_{1} \\ \vdots \\ a_{i} \\ \vdots \\ a_{n}
\end{array}
\right) \ ,
\hspace{10mm}
a_{i} \in {\mathbb R},\hspace{3mm}
i=1, \dots, n \ ,
$}
\ee
%
Notation: $\vec{a} \in \mathbb{R}^{n \times 1}$,

\medskip
\noindent
and

\medskip
\noindent
\underline{\bf Def.:}
Real-valued {\bf row vector} with $n$ 
components
%
\be
\fbox{$\displaystyle
\vec{a}^{T} := \left(
a_{1}, \dots, a_{i}, \dots, a_{n}\right) \ ,
\hspace{10mm}
a_{i} \in {\mathbb R},\hspace{3mm}
i=1, \dots, n \ ,
$}
\ee
%
Notation: $\vec{a}^{T} \in \mathbb{R}^{1 \times n}$.

\medskip
\noindent
Correspondingly, we define the $n$-component objects
%
\be
\vec{0} := \left(
\begin{array}{c}
0 \\ \vdots \\ 0 \\ \vdots \\ 0
\end{array}
\right)
\qquad\text{and}\qquad
\vec{0}^{T} := \left(
0, \dots, 0, \dots, 0\right)
\ee
%
to constitute related {\bf zero vectors}.

\medskip
\noindent
Next we define for like objects in the set $V$, i.e., either for 
$n$-component column vectors or for $n$-component row vectors, two 
simple computational operations. These are

\medskip
\noindent
\underline{\bf Def.:}
{\bf Addition} of vectors
%
\be
\lb{vekadd}
\fbox{$\displaystyle
\vec{a} + \vec{b}
:= \left(
\begin{array}{c}
a_{1} \\ \vdots \\ a_{i} \\ \vdots \\ a_{n}
\end{array}
\right)
+ \left(
\begin{array}{c}
b_{1} \\ \vdots \\ b_{i} \\ \vdots \\ b_{n}
\end{array}
\right)
= \left(
\begin{array}{c}
a_{1}+b_{1} \\ \vdots \\ a_{i}+b_{i} \\ \vdots \\
a_{n}+b_{n}
\end{array}
\right) \ ,
\hspace{10mm}
a_{i}, b_{i} \in {\mathbb R} \ ,
$}
\ee
%

\medskip
\noindent
and

\medskip
\noindent
\underline{\bf Def.:}
{\bf Rescaling} of vectors
%
\be
\lb{vekskal}
\fbox{$\displaystyle
\lambda\vec{a}
:= \left(
\begin{array}{c}
\lambda a_{1} \\ \vdots \\ \lambda a_{i} \\ \vdots \\
\lambda a_{n}
\end{array}
\right) \ ,
\hspace{10mm}
\lambda, a_{i} \in {\mathbb R} \ .
$}
\ee
%

\medskip
\noindent
The rescaling of a vector $\vec{a}$ with an arbitrary non-zero real number  $\lambda$ has the following effects:
%
\begin{itemize}
	\item $|\lambda| > 1$ --- stretching of the length of $\vec{a}$
	\item $0 < |\lambda| < 1$ --- shrinking of the length of 
	$\vec{a}$
	\item $\lambda < 0$ --- directional reversal of $\vec{a}$.
\end{itemize}
%
The notion of the length of a vector $\vec{a}$ will be made precise shortly.

\medskip
\noindent
The addition and the rescaling of $n$-component vectors satisfy the following addition and multiplication laws:

\medskip
\noindent
{\bf Computational rules for addition and rescaling of vectors}

\noindent
For vectors $\vec{a}, \vec{b}, \vec{c} \in {\mathbb R}^{n}$:
%
\begin{enumerate}
	\item $\vec{a}+\vec{b} = \vec{b}+\vec{a}$
	\hfill ({\bf commutative addition})
	\item $\vec{a}+(\vec{b}+\vec{c}) = (\vec{a}+\vec{b})+\vec{c}$
	\hfill ({\bf associative addition})
	\item $\vec{a}+\vec{0}=\vec{a}$ \hfill ({\bf addition identity element})
	\item For every $\vec{a}, \vec{b} \in {\mathbb R}^{n}$, there exists exactly one  $\vec{x} \in {\mathbb R}^{n}$ such that $\vec{a}+\vec{x}=\vec{b}$
	
{} \hfill ({\bf invertibility of addition})
	\item $(\lambda\mu)\vec{a}=\lambda(\mu\vec{a})$ with $\lambda
	\in {\mathbb R}$ \hfill ({\bf associative rescaling})
	\item $1\vec{a}=\vec{a}$ \hfill ({\bf rescaling identity element})
	\item $\lambda(\vec{a}+\vec{b}) = \lambda\vec{a}+\lambda\vec{b}$;
	
	$(\lambda+\mu)\vec{a} = \lambda\vec{a}+\mu\vec{a}$ with $\lambda,
	\mu	\in {\mathbb R}$ \hfill ({\bf distributive rescaling}).
\end{enumerate}
%

\medskip
\noindent
In conclusion of this section, we remark that every set of 
mathematical objects $V$ constructed in line with 
Eq.~(\ref{vecsp}), with an addition and a rescaling defined 
according to Eqs.~(\ref{vekadd}) and~(\ref{vekskal}), and 
satisfying the laws stated above, constitutes a {\bf linear vector 
space over Euclidian space} ${\mathbb R}^{n}$.\footnote{This is 
named after the ancient greek mathematician
\href{http://www-groups.dcs.st-and.ac.uk/~history/Biographies/Euclid.html}{Euclid of Alexandria (about 325~BC--265~BC)}.}

%%%%%%%%%%%%%%%%%%%%%%%%%%%%%%%%%%%%%%%%%%%%%%%%%%%%%%%%%%%%%%%%%%%
\section[Dimension and basis of ${\mathbb R}^{n}$]%
{Dimension and basis of ${\mathbb R}^{n}$}
\lb{sec:vekdim}
%%%%%%%%%%%%%%%%%%%%%%%%%%%%%%%%%%%%%%%%%%%%%%%%%%%%%%%%%%%%%%%%%%%
Let there be given $m$ $n$-component vectors\footnote{A slightly
shorter notation for $n$-component column vectors $\vec{a} \in 
\mathbb{R}^{n\times 1}$ is given by $\vec{a} \in \mathbb{R}^{n}$; 
likewise $\vec{a}^{T} \in \mathbb{R}^{n}$ for $n$-component row 
vectors $\vec{a}^{T} \in \mathbb{R}^{1\times n}$.} $\vec{a}_{1}, 
\ldots, \vec{a}_{i}, \ldots, \vec{a}_{m} \in {\mathbb R}^{n}$, as 
well as $m$ real numbers $\lambda_{1}, \ldots, \lambda_{i}, 
\ldots, \lambda_{m} \in {\mathbb R}$. The new $n$-component vector 
$\vec{b}$ resulting from the addition of arbitrarily rescaled 
versions of these $m$ vectors according to
%
\be
\lb{linkomb}
\fbox{$\displaystyle
\vec{b} = \lambda_{1}\vec{a}_{1}+\ldots+\lambda_{i}\vec{a}_{i}
+\ldots+\lambda_{m}\vec{a}_{m}
=: \sum_{i=1}^{m}\lambda_{i}\vec{a}_{i} \in {\mathbb R}^{n}
$}
\ee
%
is referred to as a {\bf linear combination} 
of the $m$ vectors $\vec{a}_{i},\,i=1, \ldots, m$.

\medskip
\noindent
\underline{\bf Def.:} A set of $m$ vectors $\vec{a}_{1}, \ldots,
\vec{a}_{i}, \ldots, \vec{a}_{m} \in {\mathbb R}^{n}$ is called 
{\bf linearly independent} when the condition
%
\be
\lb{linunabh}
\boldsymbol{0} \stackrel{!}{=} \lambda_{1}\vec{a}_{1}+\ldots
+\lambda_{i}\vec{a}_{i}+\ldots+\lambda_{m}\vec{a}_{m}
= \sum_{i=1}^{m}\lambda_{i}\vec{a}_{i} \ ,
\ee
%
i.e., the problem of forming the {\bf zero vector} $\boldsymbol{0} 
\in {\mathbb R}^{n}$ from a linear combination of the $m$ vectors 
$\vec{a}_{1}, \ldots, \vec{a}_{i}, \ldots, \vec{a}_{m} \in 
{\mathbb R}^{n}$, can \emph{only} be solved trivially, namely by 
$0=\lambda_{1}=\ldots=\lambda_{i}=\ldots=\lambda_{m}$. When, 
however, this condition can be solved non-trivially, with some 
$\lambda_{i} \neq 0$, then the set of $m$ vectors $\vec{a}_{1},
\ldots, \vec{a}_{i}, \ldots,\vec{a}_{m} \in {\mathbb R}^{n}$ is 
called {\bf linearly dependent}.

\medskip
\noindent
In Euclidian space ${\mathbb R}^{n}$, there is a maximum number 
$n$ (!) of vectors which can be linearly independent. This maximum 
number is referred to as the {\bf dimension of Euclidian 
space} ${\mathbb R}^{n}$. Every set of $n$ linearly independent 
vectors in Euclidian space ${\mathbb R}^{n}$ constitutes a 
possible {\bf basis of Euclidian space} ${\mathbb R}^{n}$.
If the set $\{\vec{a}_{1}, \ldots, \vec{a}_{i}, 
\ldots,\vec{a}_{n}\}$ constitutes a basis of ${\mathbb R}^{n}$, 
then every other vector $\vec{b} \in {\mathbb R}^{n}$ can be 
expressed in terms of these basis vectors by
%
\be
\vec{b} = \beta_{1}\vec{a}_{1}+\ldots+\beta_{i}\vec{a}_{i}
+\ldots+\beta_{n}\vec{a}_{n}
= \sum_{i=1}^{n}\beta_{i}\vec{a}_{i} \ .
\ee
%
The rescaling factors $\beta_{i}\in{\mathbb R}$ of the 
$\vec{a}_{i}\in{\mathbb R}^{n}$ are called the {\bf components of 
vector} $\vec{b}$ {\bf with respect to the basis}
$\{\vec{a}_{1}, \ldots, \vec{a}_{i}, \ldots, \vec{a}_{n}\}$.

\vspace{5mm}
\noindent
\underline{\bf Remark:} The $n$ {\bf unit vectors}
%
\be
\lb{kanbasis}
\vec{e}_{1} := \left(
\begin{array}{c}
1 \\ 0 \\ \vdots \\ 0
\end{array}
\right) \ , \hspace{5mm}
\vec{e}_{2} := \left(
\begin{array}{c}
0 \\ 1 \\ \vdots \\ 0
\end{array}
\right) \ , \hspace{5mm}
\dots \ , \hspace{5mm}
\vec{e}_{n} := \left(
\begin{array}{c}
0 \\ 0 \\ \vdots \\ 1
\end{array}
\right) \ , \hspace{5mm}
\ee
%
constitute the so-called {\bf canonical basis of Euclidian 
space} ${\mathbb R}^{n}$. With respect to this basis, all vectors 
$\vec{b} \in \mathbb{R}^{n}$ can be represented as a linear 
combination
%
\be
\vec{b} = \left(
\begin{array}{c}
b_{1} \\ b_{2} \\ \vdots \\ b_{n}
\end{array}
\right)
= b_{1}\vec{e}_{1}+b_{2}\vec{e}_{2}+\dots
+b_{n}\vec{e}_{n}
= \sum_{i=1}^{n}b_{i}\vec{e}_{i} \ .
\ee
%

%%%%%%%%%%%%%%%%%%%%%%%%%%%%%%%%%%%%%%%%%%%%%%%%%%%%%%%%%%%%%%%%%%%
\section[Euclidian scalar product]%
{Euclidian scalar product}
\lb{sec:vekskal}
%%%%%%%%%%%%%%%%%%%%%%%%%%%%%%%%%%%%%%%%%%%%%%%%%%%%%%%%%%%%%%%%%%%
Finally, to conclude this section, we introduce a third mathematical operation defined for vectors on $\mathbb{R}^{n}$.

\medskip
\noindent
\underline{\bf Def.:} For an $n$-component row vector $\vec{a}^{T} 
\in \mathbb{R}^{1 \times n}$ and an $n$-component column vector 
$\vec{b} \in \mathbb{R}^{n \times 1}$, the {\bf Euclidian scalar 
product}
%
\be
\lb{skalprod}
\fbox{$\displaystyle
\vec{a}^{T}\cdot\vec{b}
:= \left(
a_{1}, \ldots, a_{i}, \ldots a_{n}\right)
\left(
\begin{array}{c}
b_{1} \\ \vdots \\ b_{i} \\ \vdots \\ b_{n}
\end{array}
\right)
= a_{1}b_{1}+\ldots+a_{i}b_{i}\ldots+a_{n}b_{n}
=: \sum_{i=1}^{n}a_{i}b_{i}
$}
\ee
%
defines a mapping $f: \mathbb{R}^{1 \times n} \times \mathbb{R}^{n 
\times 1} \rightarrow \mathbb{R}$ from the product set of 
$n$-component row and column vectors to the set of real numbers. 
Note that, in contrast to the addition and the rescaling of 
$n$-component vectors, the outcome of forming a Euclidian scalar 
product between two $n$-component vectors is a \emph{single real 
number}.

\medskip
\noindent
In the context of the Euclidian scalar product, two non-zero 
vectors $\vec{a}, \vec{b} \in {\mathbb R}^{n}$ (wit $\vec{a}
\neq \vec{0} \neq \vec{b}$) are referred to as {\bf mutually 
orthogonal} when they exhibit the property that $0 = 
\vec{a}^{T}\cdot\vec{b} = \vec{b}^{T}\cdot\vec{a}$.

%\pagebreak
\medskip
\noindent
{\bf Computational rules for Euclidian scalar product of 
vectors}

\noindent
For vectors $\vec{a}, \vec{b}, \vec{c} \in {\mathbb R}^{n}$:

\begin{enumerate}
	\item $(\vec{a}+\vec{b})^{T}\cdot\vec{c}
	= \vec{a}^{T}\cdot\vec{c}+\vec{b}^{T}\cdot\vec{c}$
	\hfill ({\bf distributive scalar product})
	\item $\vec{a}^{T}\cdot\vec{b} = \vec{b}^{T}\cdot\vec{a}$
	\hfill ({\bf commutative scalar product})
	\item $(\lambda\vec{a}^{T})\cdot\vec{b} =
	\lambda(\vec{a}^{T}\cdot\vec{b})$ with $\lambda \in {\mathbb R}$
	\hfill ({\bf homogeneous scalar product})
	\item $\vec{a}^{T}\cdot\vec{a} > 0$ for all $\vec{a}
	\neq \vec{0}$ \hfill ({\bf positive definite scalar product}).
\end{enumerate}

\medskip
\noindent
Now we turn to introduce the notion of the length of an 
$n$-component vector.

\medskip
\noindent
\underline{\bf Def.:} The {\bf length} of a vector $\vec{a} \in 
\mathbb{R}^{n}$ is defined via the Euclidian scalar product as
%
\be
\fbox{$\displaystyle
|\vec{a}| :=
\sqrt{\vec{a}^{T}\cdot\vec{a}}
= \sqrt{a_{1}^{2}+\ldots+a_{i}^{2}+\ldots+a_{n}^{2}}
=: \sqrt{\sum_{i=1}^{n}a_{i}^{2}} \ .
$}
\ee
%

\medskip
\noindent
Technically one refers to the non-negative real number $|\vec{a}|$ 
as the {\bf absolute value} or the {\bf Euclidian norm} of the 
vector $\vec{a}\in \mathbb{R}^{n}$. The length of $\vec{a} \in
\mathbb{R}^{n}$ has the following properties: 
%
\begin{itemize}
	\item $|\vec{a}| \geq 0$, and $|\vec{a}|=0 \Leftrightarrow
	\vec{a}=\vec{0}$;
	\item $|\lambda\vec{a}| = |\lambda||\vec{a}|$ for $\lambda \in
	{\mathbb R}$;
	\item $|\vec{a}+\vec{b}| \leq |\vec{a}|+|\vec{b}|$
	\hfill ({\bf triangle inequality}).
\end{itemize}

\medskip
\noindent
Every non-zero vector $\vec{a} \in \mathbb{R}^{n}$, i.e., 
$|\vec{a}|>0$, can be rescaled by the reciprocal of its length. 
This procedure defines the

\medskip
\noindent
\underline{\bf Def.:}
{\bf Normalisation} of a vector $\vec{a} \in
\mathbb{R}^{n}$;
%
\be
\hat{\vec{a}} := \frac{\vec{a}}{|\vec{a}|}
\qquad\Rightarrow\qquad
|\hat{\vec{a}}| = 1 \ .
\ee
%

\medskip
\noindent
By this method one generates a vector of length $1$, i.e., a {\bf 
unit vector} $\hat{\vec{a}}$. To denote unit vectors we will 
employ the ``hat'' symbol.

\medskip
\noindent
Lastly, also by means of the Euclidian scalar product, we introduce the angle enclosed between two non-zero vectors.

\medskip
\noindent
\underline{\bf Def.:}
{\bf Angle} enclosed between $\vec{a},\vec{b} \neq \vec{0}
\in {\mathbb R}^{n}$ 
%
\be
\fbox{$\displaystyle
\cos[\varphi(\vec{a},\vec{b})]
= \frac{\vec{a}^{T}}{|\vec{a}|}\cdot
\frac{\vec{b}}{|\vec{b}|}
= \hat{\vec{a}}{}^{T}\cdot\hat{\vec{b}}
\hspace{5mm} \Rightarrow \hspace{5mm}
\varphi(\vec{a},\vec{b})
= \cos^{-1}(\hat{\vec{a}}{}^{T}\cdot\hat{\vec{b}}) \ .
$}
\ee
%
\underline{\bf Remark:} The inverse cosine function\footnote{The 
notion of on inverse function will be discussed later in 
Ch.~\ref{ch7}.} $\cos^{-1}(\ldots)$ is available on every standard 
GDC or spreadsheet.

%%%%%%%%%%%%%%%%%%%%%%%%%%%%%%%%%%%%%%%%%%%%%%%%%%%%%%%%%%%%%%%%%%%
%%%%%%%%%%%%%%%%%%%%%%%%%%%%%%%%%%%%%%%%%%%%%%%%%%%%%%%%%%%%%%%%%%%
