%%%%%%%%%%%%%%%%%%%%%%%%%%%%%%%%%%%%%%%%%%%%%%%%%%%%%%%%%%%%%%%%%%%
%  File name: ch3-engl.tex
%  Title:
%  Version: 11.09.2015 (hve)
%%%%%%%%%%%%%%%%%%%%%%%%%%%%%%%%%%%%%%%%%%%%%%%%%%%%%%%%%%%%%%%%%%%
%%%%%%%%%%%%%%%%%%%%%%%%%%%%%%%%%%%%%%%%%%%%%%%%%%%%%%%%%%%%%%%%%%%
\chapter[Systems of linear algebraic equations]{Systems of linear 
algebraic equations}
\lb{ch3}
%%%%%%%%%%%%%%%%%%%%%%%%%%%%%%%%%%%%%%%%%%%%%%%%%%%%%%%%%%%%%%%%%%%
%\hfill\hbox{\fbox{\vbox{\hsize=10cm
%Der Inhalt dieses Kapitels ist relevant f\"ur das
%Verst\"andnis der in Kapitel 5.2
%des Lehrbuchs von Schrey\"ogg und Koch (2010)~\ct{schkoc2010}
%behandelten quantitativen Themen.
%%Gleichfalls f\"ur Kapitel 10.5, 11.4 des Lehrbuchs von
%%Schmalen und Pechtl 2006~\ct{schpec2006}.
%}}}

\vspace{10mm}
\noindent
In this chapter, we turn to address a particular field of 
application of the notions of matrices and vectors, or of linear 
mappings in general.

%%%%%%%%%%%%%%%%%%%%%%%%%%%%%%%%%%%%%%%%%%%%%%%%%%%%%%%%%%%%%%%%%%%
\section[Basic concepts]%
{Basic concepts}
\lb{sec:lgsgrund}
%%%%%%%%%%%%%%%%%%%%%%%%%%%%%%%%%%%%%%%%%%%%%%%%%%%%%%%%%%%%%%%%%%%
Let us begin with a system of $m \in \mathbb{N}$ \emph{linear} 
algebraic equations, wherein every single equation can be 
understood to constitute a {\bf constraint} on the range of values 
of $n \in \mathbb{N}$ variables $x_{1}, %\ldots, x_{j},
\ldots, x_{n} \in \mathbb{R}$. The objective is to determine all 
possible values of $x_{1}, \ldots, x_{n} \in \mathbb{R}$ which 
satisfy these constraints simultaneously. Problems of this kind, 
namely {\bf systems of linear algebraic equations}, are often 
represented in the form

\medskip
\noindent
$\bullet$ Representation~1:\\[-7mm]
%
\bea
a_{11}x_{1}+\ldots+a_{1j}x_{j}+\ldots+a_{1n}x_{n} & = & b_{1} 
\nonumber \\
%
 & \vdots & \nonumber \\
%
a_{i1}x_{1}+\ldots+a_{ij}x_{j}+\ldots+a_{in}x_{n} & = & b_{i} \\
%
 & \vdots & \nonumber \\
%
a_{m1}x_{1}+\ldots+a_{mj}x_{j}+\ldots+a_{mn}x_{n} & = & b_{m}
\nonumber \ .
\eea
%
Depending on how the natural numbers $m$ and $n$ relate to one 
another, systems of linear algebraic equations can be classified 
as follows:\\[-7mm]
%
\begin{itemize}
\item $m < n$: fewer equations than variables; the linear system 
is {\bf under-determined},\\[-7mm]

\item $m = n$: same number of equations as variables;
the linear system is {\bf well-determined},\\[-7mm]

\item $m > n$: more equations than variables; the linear system is 
{\bf over-determined}.\\[-7mm]
\end{itemize}
%
A more compact representation of a linear system of format
$(m \times n)$ is given by

\medskip
\noindent
$\bullet$ Representation~2:
%
\be
\fbox{$\displaystyle
\mathbf{A}\vec{x} =
\left(\begin{array}{ccccc}
a_{11} & \ldots & a_{1j} & \ldots & a_{1n} \\
\vdots & \ddots & \vdots & \ddots & \vdots \\
a_{i1} & \ldots & a_{ij} & \ldots & a_{in} \\
\vdots & \ddots & \vdots & \ddots & \vdots \\
a_{m1} & \ldots & a_{mj} & \ldots & a_{mn}
\end{array}\right)
\left(\begin{array}{c}
x_{1} \\
\vdots \\
x_{j} \\
\vdots \\
x_{n}
\end{array}\right)
= \left(
\begin{array}{c}
b_{1} \\
\vdots \\
b_{i} \\
\vdots \\
b_{m}
\end{array}\right)
= \vec{b} \ .
$}
\ee
%
The mathematical objects employed in this variant of a linear 
system are as follows: $\mathbf{A}$ takes the central role of the 
{\bf coefficient matrix} of the linear system, of format $(m 
\times n)$, $\vec{x}$ is its {\bf variable vector}, of format $(n 
\times 1)$, and, lastly, $\vec{b}$ is its {\bf image vector}, of 
format $(m \times 1)$.

\medskip
\noindent
When dealing with systems of linear algebraic equations in the 
form of Representation 2, i.e. $\mathbf{A}\vec{x} = \vec{b}$, the 
main question to be answered is:

\medskip
\noindent
\underline{\bf Question:}
For given {\bf coefficient matrix} $\mathbf{A}$ and {\bf image 
vector} $\vec{b}$, can we find a {\bf variable vector} $\vec{x}$ 
that $\mathbf{A}$ maps onto $\vec{b}$?

\medskip
\noindent
In a sense this describes the inversion of the photographic 
process we had previously referred to: we \emph{have} given the 
camera and we already \emph{know} the image, but we have yet to 
find a matching object. Remarkably, to address this issue, we can 
fall back on a simple algorithmic method due to the German 
mathematician and astronomer 
\href{http://www-groups.dcs.st-and.ac.uk/~history/Biographies/Gauss.html}{Carl Friedrich Gau\ss\ (1777--1855)}.

%%%%%%%%%%%%%%%%%%%%%%%%%%%%%%%%%%%%%%%%%%%%%%%%%%%%%%%%%%%%%%%%%%%
\section[Gau\ss ian elimination]%
{Gau\ss ian elimination}
\lb{sec:lgsgauss}
%%%%%%%%%%%%%%%%%%%%%%%%%%%%%%%%%%%%%%%%%%%%%%%%%%%%%%%%%%%%%%%%%%%
The algorithmic solution technique developed by Gau\ss\ is based 
on the insight that the solution set of a {\bf linear system} of 
$m$ algebraic equations for $n$ real-valued variables, i.e.
%
\be
\fbox{$\displaystyle
\mathbf{A}\vec{x} = \vec{b} \ ,
$}
\ee
%
remains unchanged under the following algebraic {\bf equivalence 
transformations} of the linear system:
%
\begin{enumerate}
\item changing the order amongst the equations,
\item multiplication of any equation by a non-zero real number $c \neq 0$,
\item addition of a multiple of one equation to another equation,
\item changing the order amongst the equations.
\end{enumerate}
%
Specifically, this implies that we may manipulate a given linear 
system by means of these four different kinds of equivalence 
transformations without ever changing its identity. In concrete 
cases, however, one should not apply these equivalence 
transformations at random but rather follow a target oriented 
strategy. This is what Gau\ss ian elimination can provide.

\medskip
\noindent
\underline{\bf Target:} To cast the {\bf augmented coefficient 
matrix} $(\mathbf{A}|\vec{b})$, i.e., the array
%
\be
\begin{array}{ccccc|c}
a_{11} & \ldots & a_{1j} & \ldots & a_{1n} & b_{1} \\
\vdots & \ddots & \vdots & \ddots & \vdots & \vdots \\
a_{i1} & \ldots & a_{ij} & \ldots & a_{in} & b_{i} \\
\vdots & \ddots & \vdots & \ddots & \vdots & \vdots \\
a_{m1} & \ldots & a_{mj} & \ldots & a_{mn} & b_{m}
\end{array} \ ,
\ee
%
when possible, into {\bf upper triangular form}
%
\be
\begin{array}{ccccc|c}
1 & \ldots & \tilde{a}_{1j} & \ldots & \tilde{a}_{1n} & \tilde{b}_{1} \\
\vdots & \ddots & \vdots & \ddots & \vdots & \vdots \\
0 & \ldots & \tilde{a}_{ij} & \ldots & \tilde{a}_{in} & \tilde{b}_{i} \\
\vdots & \ddots & \vdots & \ddots & \vdots & \vdots \\
0 & \ldots & 0 & \ldots & \tilde{a}_{mn} & \tilde{b}_{m}
\end{array} \ ,
\ee
%
by means of the four kinds of equivalence transformations such 
that the resultant simpler final linear system may easily be 
solved using {\bf backward substitution}.

\medskip
\noindent
Three exclusive cases of possible {\bf solution content} for a 
given system of linear algebraic equations do exist. The linear 
system may possess either
%
\begin{enumerate}
\item \emph{no solution} at all, or

\item a \emph{unique solution}, or

\item \emph{multiple solutions}.
\end{enumerate}
%
\underline {\bf Remark:} Linear systems that are under-determined, 
i.e., when $m < n$, can \emph{never} be solved uniquely due to the 
fact that in such a case there not exist enough equations to 
constrain the values of \emph{all} of the $n$ variables.

\medskip
\noindent
\underline {\bf GDC:} For a stored augmented coefficient matrix 
${\tt [A]}$ of format $(m \times n+1)$, associated with a given  
$(m \times n)$ linear system, select mode {\tt MATRIX} 
$\rightarrow$ {\tt MATH} and then call the function ${\tt 
rref([A])}$. It is possible that backward substitution needs to be 
employed to obtain the final solution.

\medskip
\noindent
For completeness, we want to turn briefly to the issue of 
solvability of a system of linear algebraic equations. To this 
end, we need to introduce the notion of the rank of a matrix.

%%%%%%%%%%%%%%%%%%%%%%%%%%%%%%%%%%%%%%%%%%%%%%%%%%%%%%%%%%%%%%%%%%%
\section[Rank of a matrix]%
{Rank of a matrix}
\lb{sec:lgsrang}
%%%%%%%%%%%%%%%%%%%%%%%%%%%%%%%%%%%%%%%%%%%%%%%%%%%%%%%%%%%%%%%%%%%

\noindent
\underline{\bf Def.:}
A real-valued matrix $\mathbf{A} \in \mathbb{R}^{m \times n}$ 
possesses the {\bf rank}
%
\be
\fbox{$\displaystyle
\text{rank}(\mathbf{A}) = r \ , \qquad
r \leq \text{min}\{m,n\}
$}
\ee
%
if and only if $r$ is the {\bf maximum number} of row resp.~column 
vectors of $\mathbf{A}$ which are linearly independent. Clearly, 
$r$ can only be as large as the smaller of the numbers $m$ and $n$ 
that determine the format of $\mathbf{A}$.

\medskip
\noindent
For {\bf quadratic matrices} $\mathbf{A} \in
\mathbb{R}^{n \times n}$, there is available a more elegant 
measure to determine its rank. This (in the present case 
real-valued) measure is referred to as the {\bf determinant} of 
matrix $\mathbf{A}$, $\det(\mathbf{A})$, and is defined as follows.

\medskip
\noindent
\underline{\bf Def.:}
%
\begin{itemize}
\item[(i)]~When $\mathbf{A} \in \mathbb{R}^{2 \times 2}$,
its {\bf determinant} is given by
%
\be
\det(\mathbf{A})
:= \left|\begin{array}{cc}
   	a_{11} & a_{12} \\
   	a_{21} & a_{22}
	\end{array}\right|
:= a_{11}a_{22} - a_{12}a_{21} \ ,
\ee
%
i.e. the difference between the products of $\mathbf{A}$'s 
on-diagonal elements and $\mathbf{A}$'s off-diagonal elements.

\item[(ii)]~When $\mathbf{A} \in \mathbb{R}^{3 \times 3}$, the 
definition of $\mathbf{A}$'s {\bf determinant} is more complex. In 
that case it is given by
%
\bea
\det(\mathbf{A})
& := & \left|\begin{array}{ccc}
   	a_{11} & a_{12} & a_{13} \\
   	a_{21} & a_{22} & a_{23} \\
   	a_{31} & a_{32} & a_{33}
	\end{array}\right| \nonumber \\
	%
& := & a_{11}(a_{22}a_{33}-a_{32}a_{23})
+ a_{21}(a_{32}a_{13}-a_{12}a_{33})
+ a_{31}(a_{12}a_{23}-a_{22}a_{13}) \ .
\eea
%
Observe, term by term, the cyclic permutation of the first index 
of the elements $a_{ij}$ according to the rule $1 \rightarrow 2 
\rightarrow 3 \rightarrow 1$.

\item[(iii)]~Finally, for the (slightly involved) definition of 
the {\bf determinant} of a higher-dimensional matrix $\mathbf{A} 
\in \mathbb{R}^{n \times n}$, please refer to the literature; e.g. 
Bronstein \emph{et al} (2005)~\ct[p~267]{broetal2005}.
\end{itemize}
%

\medskip
\noindent
To determine the rank of a given quadratic matrix $\mathbf{A}
\in \mathbb{R}^{n \times n}$, one now installs the following 
criteria: $\text{rank}(\mathbf{A}) = r = n$, if $\det(\mathbf{A}) 
\neq 0$, and $\text{rank}(\mathbf{A}) = r < n$, if 
$\det(\mathbf{A}) = 0$. In the first case, $\mathbf{A}$ is 
referred to as {\bf regular}, in the second as {\bf singular}. For 
quadratic matrices $\mathbf{A}$ that are singular, 
$\text{rank}(\mathbf{A}) = r$ (with $r < n$) is given by the 
number $r$ of rows (or columns) of the largest possible non-zero 
subdeterminant of $\mathbf{A}$.

\medskip
\noindent
\underline {\bf GDC:} For a stored quadratic matrix ${\tt [A]}$, 
select mode {\tt MATRIX} $\rightarrow$ {\tt MATH} and obtain its 
determinant by calling the function ${\tt det([A])}$.

%%%%%%%%%%%%%%%%%%%%%%%%%%%%%%%%%%%%%%%%%%%%%%%%%%%%%%%%%%%%%%%%%%%
\section[Solution criteria]%
{Criteria for solving systems of linear algebraic equations}
\lb{sec:lgsloes}
%%%%%%%%%%%%%%%%%%%%%%%%%%%%%%%%%%%%%%%%%%%%%%%%%%%%%%%%%%%%%%%%%%%
Making use of the concept of the {\bf rank} of a real-valued 
matrix $\mathbf{A} \in \mathbb{R}^{m \times n}$, we can now 
summarise the solution content of a specific system of linear 
algebraic equations of format $(m \times n)$ in a table. For given 
linear system
%
$$
\mathbf{A}\vec{x} = \vec{b} \ ,
$$
%
with coefficient matrix $\mathbf{A} \in \mathbb{R}^{m \times n}$, 
variable vector $\vec{x} \in \mathbb{R}^{n \times 1}$ and image 
vector $\vec{b} \in \mathbb{R}^{m \times 1}$,
%as indicated in the table below,
there exist(s)
%
\begin{center}
	\begin{tabular}[h]{c|c|c}
		\hline\hline
		 & & \\
		 & $\vec{b} \neq \vec{0}$ &
		 $\vec{b} = \vec{0}$ \\
		 & & \\
		\hline
		 & & \\
		1. $\text{rank}(\mathbf{A}) \neq \text{rank}(\mathbf{A}|\vec{b})$ &
		no solution & -------- \\
		 & & \\
		\hline
		 & & \\
		2. $\text{rank}(\mathbf{A}) = \text{rank}(\mathbf{A}|\vec{b}) = r$ &
		& \\
		 & & \\
		(a) \quad $r = n$ & a unique & $\vec{x} = \vec{0}$ \\
		 & solution & \\
		 & & \\
		(b) \quad $r < n$ & multiple & multiple \\
		 & solutions: & solutions: \\
		 & $n-r$ free &  $n-r$ free \\
		 & parameters & parameters \\
		 & & \\
		\hline\hline
	 \end{tabular}
\end{center}
%

\noindent
$(\mathbf{A}|\vec{b})$ here denotes the augmented coefficient 
matrix.

\medskip
\noindent
Next we discuss a particularly useful property of \emph{regular} 
quadratic matrices.

%%%%%%%%%%%%%%%%%%%%%%%%%%%%%%%%%%%%%%%%%%%%%%%%%%%%%%%%%%%%%%%%%%%
\section[Inverse of a regular $(n\times n)$-matrix]%
{Inverse of a regular $(n\times n)$-matrix}
\lb{sec:lgsinv}
%%%%%%%%%%%%%%%%%%%%%%%%%%%%%%%%%%%%%%%%%%%%%%%%%%%%%%%%%%%%%%%%%%%

\noindent
\underline{\bf Def.:}
Let a real-valued quadratic matrix $\mathbf{A} \in
\mathbb{R}^{n \times n}$ be {\bf regular}, i.e., $\det(\mathbf{A}) 
\in \mathbb{R}\backslash\{0\}$. Then there exists an {\bf inverse 
matrix} $\mathbf{A}^{-1}$ to $\mathbf{A}$ defined by the 
characterising properties
%
\be
\fbox{$\displaystyle
\mathbf{A}^{-1}\mathbf{A} = \mathbf{A}\mathbf{A}^{-1}
= \mathbf{1} \ .
$}
\ee
%
Here $\mathbf{1}$ denotes the $\boldsymbol{(n \times n)}${\bf -unit matrix} [cf.\ Eq.~(\ref{einmatr})].

\medskip
\noindent
When a computational device is not at hand, the inverse matrix 
$\mathbf{A}^{-1}$ of a regular quadratic matrix $\mathbf{A}$ can 
be obtained by solving the matrix-valued linear system
%
\be
\mathbf{A}\mathbf{X} \stackrel{!}{=} \mathbf{1}
\ee
%
for the unknown matrix $\mathbf{X}$ by means of {\bf simultaneous 
Gau\ss ian elimination}.

\medskip
\noindent
\underline{\bf GDC:} For a stored quadratic matrix ${\tt [A]}$, 
its inverse matrix can be simply obtained as ${\tt [A]}^{-1}$, 
where the $x^{-1}$ function key needs to be used.

\pagebreak
\medskip
\noindent
{\bf Computational rules for the inverse operation}

\nopagebreak
\noindent
For $\mathbf{A},\mathbf{B} \in \mathbb{R}^{n \times n}$,
with $\det(\mathbf{A}) \neq 0 \neq \det(\mathbf{B})$, it holds that

\begin{enumerate}
\item $(\mathbf{A}^{-1})^{-1} = \mathbf{A}$
\item $(\mathbf{A}\mathbf{B})^{-1}
= \mathbf{B}^{-1}\mathbf{A}^{-1}$
\item $(\mathbf{A}^{T})^{-1} = (\mathbf{A}^{-1})^{T}$
\item $\displaystyle (\lambda\mathbf{A})^{-1}
= \frac{1}{\lambda}\,\mathbf{A}^{-1}$.
\end{enumerate}

\medskip
\noindent
The special interest in applications in the concept of {\bf 
inverse matrices} arises for the following reason. Consider given 
a well-determined linear system
\[
\mathbf{A}\vec{x}=\vec{b} \ ,
\]
with \emph{regular} quadratic coefficient matrix $\mathbf{A} 
\in \mathbb{R}^{n \times n}$, i.e., $\det(\mathbf{A}) \neq 0$. 
Then, for $\mathbf{A}$, there exists an inverse matrix 
$\mathbf{A}^{-1}$. Matrix-multiplying both sides of the equation 
above \emph{from the left~(!)} by the inverse $\mathbf{A}^{-1}$, 
results in
%
\be
\underbrace{\mathbf{A}^{-1}(\mathbf{A}\vec{x})
= (\mathbf{A}^{-1}\mathbf{A})\vec{x}
= \mathbf{1}\vec{x}
= \vec{x}}_{\text{left-hand side}}
= \underbrace{\mathbf{A}^{-1}\vec{b}}_{\text{right-hand side}} \ .
\ee
%
In this case, the \emph{unique solution~(!)} $\vec{x} = 
\mathbf{A}^{-1}\vec{b}$ of the linear system arises simply from 
matrix multiplication of the image vector $\vec{b}$ by the inverse 
matrix of $\mathbf{A}$. (Of course, it might actually require a 
bit of computational work to determine $\mathbf{A}^{-1}$.)

%%%%%%%%%%%%%%%%%%%%%%%%%%%%%%%%%%%%%%%%%%%%%%%%%%%%%%%%%%%%%%%%%%%
\section[Outlook]{Outlook}
\lb{sec:lgsausblick}
%%%%%%%%%%%%%%%%%%%%%%%%%%%%%%%%%%%%%%%%%%%%%%%%%%%%%%%%%%%%%%%%%%%
There are a number of exciting advanced topics in {\bf Linear 
Algebra}. Amongst them one finds the concept of the characteristic 
{\bf eigenvalues} and associated {\bf eigenvectors} of {\bf 
quadratic matrices}, which has particularly high relevance in 
practical applications. The question to be answered here is the 
following: for given real-valued quadratic matrix $\mathbf{A} \in 
\mathbb{R}^{n \times n}$, do there exist real numbers $\lambda_{n} 
\in \mathbb{R}$ and real-valued vectors $\vec{v}_{n} \in 
\mathbb{R}^{n \times 1}$ which satisfy the condition
%
\be
\lb{eigenveq1}
\mathbf{A}\vec{v}_{n} \stackrel{!}{=} \lambda_{n}\vec{v}_{n} \ ?
\ee
%
Put differently: for which vectors $\vec{v}_{n} \in \mathbb{R}^{n 
\times 1}$ does their mapping by a quadratic matrix $\mathbf{A} 
\in \mathbb{R}^{n \times n}$ amount to simple rescalings by real 
numbers $\lambda_{n} \in \mathbb{R}$?

\medskip
\noindent
By re-arranging, Eq.~(\ref{eigenveq1}) can be recast into the form
%
\be
\lb{eigenveq2}
\boldsymbol{0} \stackrel{!}{=}
\left(\mathbf{A}-\lambda_{n}\boldsymbol{1}\right)\vec{v}_{n} \ ,
\ee
%
with $\boldsymbol{1}$ an $(n \times n)$-unit matrix
[cf.\ Eq.~(\ref{einmatr})] and $\boldsymbol{0}$ an $n$-component 
zero vector. This condition corresponds to a homogeneous system of 
linear algebraic equations of format $(n \times n)$. Non-trivial 
solutions $\vec{v}_{n} \neq \boldsymbol{0}$ to this system exist 
provided that the so-called {\bf characteristic equation}
%
\be
0 \stackrel{!}{=} 
\det\left(\mathbf{A}-\lambda_{n}\boldsymbol{1}\right) \ ,
\ee
%
a polynomial of degree $n$ (cf. Sec.~\ref{subsec:polynomials}), 
allows for real-valued roots $\lambda_{n} \in \mathbb{R}$. Note 
that \emph{symmetric} quadratic matrices (cf. 
Sec.~\ref{sec:matrech}) possess exclusively real-valued 
eigenvalues $\lambda_{n}$. When these eigenvalues turn out to be 
all \emph{different}, then the associated 
eigenvectors~$\vec{v}_{n}$ prove to be mutually orthogonal.

\medskip
\noindent
Knowledge of the spectrum of {\bf eigenvalues} $\lambda_{n} \in 
\mathbb{R}$ and associated {\bf eigenvectors} $\vec{v}_{n} \in 
\mathbb{R}^{n \times 1}$ of a real-valued matrix
$\mathbf{A} \in \mathbb{R}^{n \times n}$ provides the basis of a 
transformation of $\mathbf{A}$ to its {\bf diagonal 
form}~$\mathbf{A}_{\lambda_{n}}$, thus yielding a diagonal matrix 
which features the eigenvalues $\lambda_{n}$ as its on-diagonal 
elements; cf. Leon (2009)~\ct{leo2009}.

\medskip
\noindent
Amongst other examples, the concept of eigenvalues and 
eigenvectors of quadratic real-valued matrices plays a special 
role in {\bf Statistics}, in the context of exploratory {\bf 
principal component analyses} of multivariate data sets, where the 
objective is to identify dominant intrinsic structures; cf. Hair 
\emph{et al} (2010)~\ct[Ch.~3]{haietal2010} and 
Ref.~\ct[App.~A]{hve2015}.

%%%%%%%%%%%%%%%%%%%%%%%%%%%%%%%%%%%%%%%%%%%%%%%%%%%%%%%%%%%%%%%%%%%
%%%%%%%%%%%%%%%%%%%%%%%%%%%%%%%%%%%%%%%%%%%%%%%%%%%%%%%%%%%%%%%%%%%
