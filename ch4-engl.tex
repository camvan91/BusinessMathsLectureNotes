%%%%%%%%%%%%%%%%%%%%%%%%%%%%%%%%%%%%%%%%%%%%%%%%%%%%%%%%%%%%%%%%%%%
%  File name: ch4-engl.tex
%  Title:
%  Version: 11.09.2015 (hve)
%%%%%%%%%%%%%%%%%%%%%%%%%%%%%%%%%%%%%%%%%%%%%%%%%%%%%%%%%%%%%%%%%%%
%%%%%%%%%%%%%%%%%%%%%%%%%%%%%%%%%%%%%%%%%%%%%%%%%%%%%%%%%%%%%%%%%%%
\chapter[Leontief's 
input--output matrix model]{Leontief's stationary
input--output matrix model}
\lb{ch4}
%%%%%%%%%%%%%%%%%%%%%%%%%%%%%%%%%%%%%%%%%%%%%%%%%%%%%%%%%%%%%%%%%%%
%\hfill\hbox{\fbox{\vbox{\hsize=10cm
%Der Inhalt dieses Kapitels ist relevant f\"ur das
%Verst\"andnis der in Kapitel 5.2 des Lehrbuchs von Schrey\"ogg
%und Koch (2010)~\ct{schkoc2010} behandelten quantitativen Themen.
%%Gleichfalls f\"ur Kapitel 10.5 des Lehrbuchs von Schmalen und
%%Pechtl 2006~\ct{schpec2006}.
%}}}

\vspace{10mm}
\noindent
We now turn to discuss some specific applications of {\bf Linear 
Algebra} in economic theory. To begin with, let us consider 
quantitative aspects of the exchange of goods between a certain 
number of {\bf economic agents}. We here aim at a simplified 
abstract description of real economic processes.

%%%%%%%%%%%%%%%%%%%%%%%%%%%%%%%%%%%%%%%%%%%%%%%%%%%%%%%%%%%%%%%%%%%
\section[General considerations]%
{General considerations}
\lb{sec:leongrund}
%%%%%%%%%%%%%%%%%%%%%%%%%%%%%%%%%%%%%%%%%%%%%%%%%%%%%%%%%%%%%%%%%%%
The quantitative model to be described in the following is due to 
the Russian economist 
\href{http://en.wikipedia.org/wiki/Leontief}{Wassily Wassilyovich 
Leontief (1905--1999)}, cf. Leontief (1936)~\ct{leo1936}, for 
which, besides other important contributions, he was awarded the 
1973 
\href{http://www.nobelprize.org/nobel_prizes/economics/laureates/1973/}{Sveriges Riksbank Prize in Economic Sciences in Memory of 
Alfred Nobel}. 

\medskip
\noindent
Suppose given an economic system consisting of $n \in 
\mathbb{N}$~{\bf interdependent economic agents} exchanging 
between them the goods they produce. For simplicity we want to 
\emph{assume} that every one of these {\bf economic agents} 
represents the production of a \emph{single} good only. Presently 
we intend to monitor the flow of goods in this simple economic 
system during a specified {\bf reference period of time}. The 
total numbers of the $n$~goods leaving the production sector of 
this model constitute the {\bf OUTPUT quantities}. The {\bf INPUT 
quantities} to the production sector are twofold. On the one hand, 
there are {\bf exogenous} INPUT quantities which we take to be 
given by $m \in \mathbb{N}$ different kinds of external {\bf 
resources} needed in differing proportions to produce the 
$n$~goods. On the other hand, due to their mutual interdependence, 
some of the {\bf economic agents} require {\bf goods made by their 
neighbours} to be able to produce their own goods; these 
constitute the {\bf endogenous} INPUT quantities of the system. 
Likewise, the production sector's total OUTPUT during the chosen 
reference period of the $n$~goods can be viewed to flow through 
one of \emph{two} separate channels: (i)~the {\bf exogenous} 
channel linking the production sector to {\bf external consumers} 
representing an open market, and (ii)~the {\bf endogenous} channel 
linking the {\bf economic agents} to their {\bf neighbours} (thus 
respresenting their interdependencies). It is supposed that 
momentum is injected into the economic system, triggering the flow 
of goods between the different actors, by the prospect of {\bf 
increasing the value} of the INPUT quantities, in line with the 
notion of the economic {\bf value chain}.

%\pagebreak
\medskip
\noindent
Leontief's model is based on the following three elementary

\medskip
\noindent
{\bf Assumptions}:
%
\begin{enumerate}
\item For all goods involved the functional relationship between 
INPUT and OUTPUT quantities be of a {\bf linear nature} [cf.\ 
Eq.~(\ref{lin})].

\item The proportions of ``INPUT quantities to OUTPUT quantities'' be {\bf constant} over the reference period of time considered; the flows of goods are thus considered to be {\bf stationary}.

\item {\bf Economic equilibrium} obtains during the reference period of time: the numbers of goods then supplied equal the numbers of goods then demanded.
\end{enumerate}
%
The mathematical formulation of Leontief's quantitative
model employs the following

\medskip
\noindent
{\bf Vector- and matrix-valued quantities}:
%\footnote{We use the letter ``u'' to represent adequate units of 
%measurement.}
%
\begin{enumerate}

\item $\vec{q}$ --- {\bf total output vector}
$\in \mathbb{R}^{n \times 1}$, components $q_{i} \geq 
0~\text{units}$
\hfill (dim: $\text{units}$)

\item $\vec{y}$ --- {\bf final demand vector}
$\in \mathbb{R}^{n \times 1}$, components $y_{i} \geq 
0~\text{units}$
\hfill (dim: $\text{units}$)

\item $\mathbf{P}$ --- {\bf input--output matrix}
$\in \mathbb{R}^{n \times n}$, components $P_{ij} \geq 0$
\hfill (dim: 1)

\item $(\mathbf{1}-\mathbf{P})$ --- {\bf technology matrix}
$\in \mathbb{R}^{n \times n}$, regular, hence, invertible
\hfill (dim: 1)

\item $(\mathbf{1}-\mathbf{P})^{-1}$ --- {\bf total demand matrix}
$\in \mathbb{R}^{n \times n}$
\hfill (dim: 1)

\item $\vec{v}$ --- {\bf resource vector}
$\in \mathbb{R}^{m \times 1}$, components $v_{i} \geq 
0~\text{units}$
\hfill (dim: $\text{units}$)

\item $\mathbf{R}$ --- {\bf resource consumption matrix}
$\in \mathbb{R}^{m \times n}$, components $R_{ij} \geq 0$,
\hfill (dim: 1)

\end{enumerate}
%
where $\mathbf{1}$ denotes the {\bf $\boldsymbol{(n \times 
n)}$-unit matrix} [cf.\ Eq.~(\ref{einmatr})]. Note that the 
components of all the vectors involved, as well as of the 
input--output matrix and of the resource consumption matrix, can 
assume \emph{non-negative values (!)} only.

%%%%%%%%%%%%%%%%%%%%%%%%%%%%%%%%%%%%%%%%%%%%%%%%%%%%%%%%%%%%%%%%%%%
\section[Input--output matrix and resource consumption matrix]%
{Input--output matrix and resource consumption matrix}
\lb{sec:inoutmat}
%%%%%%%%%%%%%%%%%%%%%%%%%%%%%%%%%%%%%%%%%%%%%%%%%%%%%%%%%%%%%%%%%%%
We now turn to provide the definition of the two central matrix-valued quantities in Leontief's model. We will also highlight their main characteristic features.

%------------------------------------------------------------------
\subsection{Input--output matrix}
%------------------------------------------------------------------
Suppose the {\bf reference period of time} has ended for the 
economic system in question, i.e. the stationary {\bf flows of 
goods} have stopped eventually. We now want to take stock of the 
{\bf numbers of goods} that have been delivered by each of the 
$n$~{\bf economic agents} in the system. Say that during the 
reference period considered, agent~$1$ delivered of their good the 
number $n_{11}$ to themselves, the number $n_{12}$ to agent~$2$, 
the number $n_{13}$ to agent~$3$, and so on, and, lastly, the 
number $n_{1n}$ to agent~$n$. The number delivered by agent~$1$ to 
external consumers shall be denoted by $y_{1}$. Since in this 
model a good produced \emph{cannot} all of a sudden disappear 
again, and since by Assumption~3 above the number of goods 
supplied must be equal to the number of goods demanded, we find 
that for the total output of agent~$1$ it holds that 
$q_{1}:=n_{11} + \ldots + n_{1j} + \ldots + n_{1n} + y_{1}$. 
Analogous relations hold for the total output $q_{2}$, $q_{3}$, 
\ldots, $q_{n}$ of each of the remaining $n-1$ agents. We thus 
obtain the intermediate result
%
\begin{eqnarray}
q_{1} & = & n_{11} + \ldots + n_{1j} + \ldots + n_{1n} + y_{1} > 0
\\
 & \vdots & \nonumber \\
q_{i} & = & n_{i1} + \ldots + n_{ij} + \ldots + n_{in} + y_{i} > 0
\\
 & \vdots & \nonumber \\
q_{n} & = & n_{n1} + \ldots + n_{nj} + \ldots + n_{nn} + y_{n} > 0
\ .
\end{eqnarray}
%
This simple system of {\bf balance equations} can be summarised in terms of a standard {\bf input--output table} as follows:
%
\begin{center}
    \begin{tabular}[h]{c|ccccc|c|c}
    \hline\hline
    [Values in $\text{units}$] & agent~$1$ & $\cdots$ & agent~$j$ & $\cdots$ & agent~$n$ & external consumers & $\Sigma$: total output  \\
    \hline
    agent~$1$ & $n_{11}$ & $\ldots$ & $n_{1j}$ & $\ldots$ & $n_{1n}$ & $y_{1}$ & $q_{1}$ \\
    $\vdots$ & $\vdots$ & $\ddots$ & $\vdots$ & $\ddots$ & $\vdots$ & $\vdots$ & $\vdots$ \\
    agent~$i$ & $n_{i1}$ & \ldots & $n_{ij}$ & \ldots & $n_{in}$ & $y_{i}$ & $q_{i}$ \\
    $\vdots$ & $\vdots$ & $\ddots$ & $\vdots$ & $\ddots$ & $\vdots$ & $\vdots$ & $\vdots$ \\
    agent~$n$ & $n_{n1}$ & \ldots & $n_{nj}$ & \ldots & $n_{nn}$ & $y_{n}$ & $q_{n}$ \\
    \hline\hline
    \end{tabular}
\end{center}
%
The first column of this table lists all the $n$ different {\bf 
sources of flows of goods} (or suppliers of goods), while its 
first row shows the $n+1$ different {\bf sinks of flows of goods} 
(or consumers of goods). The last column contains the total output 
of each of the $n$ agents in the {\bf reference period of time}.

\medskip
\noindent
Next we compute for each of the $n$ agents the respective values 
of the \emph{non-negative ratios}
%
\be
P_{ij} := \frac{\text{INPUT\ (in\ units)\ of\ agent\ $i$\ for
\ agent $j$\ (during\ reference\ period)}}{
\text{OUTPUT\ (in\ units)\ of\ agent\ $j$\ (during\ reference
\ period)}} \ ,
\ee
%
or, employing a compact and economical index 
notation,\footnote{Note that the normalisation quantities in these 
ratios $P_{ij}$ are given by the total output $q_{j}$ of the 
receiving agent~$j$ and \emph{not} by the total output $q_{i}$ of 
the supplying agent~$i$. In the latter case the $P_{ij}$ would 
represent percentages of the total output $q_{i}$.}
%
\be
\fbox{$\displaystyle
P_{ij} := \frac{n_{ij}}{q_{j}} \ ,
$}
\ee
%
with $i,j = 1, \ldots, n$. These $n \times n = n^{2}$ different 
ratios may be naturally viewed as the elements of a quadratic 
matrix $\mathbf{P}$ of format $(n \times n)$. In general, this 
matrix is given by
%
\be
\fbox{$\displaystyle
\mathbf{P} =
\left(\begin{array}{ccccc}
\frac{n_{11}}{n_{11}+\ldots+n_{1j}+\ldots+n_{1n}+y_{1}} &
\ldots &
\frac{n_{1j}}{n_{j1}+\ldots+n_{jj}+\ldots+n_{jn}+y_{j}} &
\ldots &
\frac{n_{1n}}{n_{n1}+\ldots+n_{nj}+\ldots+n_{nn}+y_{n}} \\
\vdots & \ddots & \vdots & \ddots & \vdots \\
\frac{n_{i1}}{n_{11}+\ldots+n_{1j}+\ldots+n_{1n}+y_{1}} &
\ldots &
\frac{n_{ij}}{n_{j1}+\ldots+n_{jj}+\ldots+n_{jn}+y_{j}} &
\ldots &
\frac{n_{in}}{n_{n1}+\ldots+n_{nj}+\ldots+n_{nn}+y_{n}} \\
\vdots & \ddots & \vdots & \ddots & \vdots \\
\frac{n_{n1}}{n_{11}+\ldots+n_{1j}+\ldots+n_{1n}+y_{1}} &
\ldots &
\frac{n_{nj}}{n_{j1}+\ldots+n_{jj}+\ldots+n_{jn}+y_{j}} &
\ldots &
\frac{n_{nn}}{n_{n1}+\ldots+n_{nj}+\ldots+n_{nn}+y_{n}}
\end{array}\right) \ ,
$}
\ee
%
and is referred to as Leontief's {\bf input--output matrix} of the 
stationary economic system under investigation.

\medskip
\noindent
For the very simple case with just $n=3$ producing agents, the 
input--output matrix reduces to
%
\[
\displaystyle
\mathbf{P} =
\left(\begin{array}{ccc}
\frac{n_{11}}{n_{11}+n_{12}+n_{13}+y_{1}} &
\frac{n_{12}}{n_{21}+n_{22}+n_{23}+y_{2}} &
\frac{n_{13}}{n_{31}+n_{32}+n_{33}+y_{3}} \\
\frac{n_{21}}{n_{11}+n_{12}+n_{13}+y_{1}} &
\frac{n_{22}}{n_{21}+n_{22}+n_{23}+y_{2}} &
\frac{n_{23}}{n_{31}+n_{32}+n_{33}+y_{3}} \\
\frac{n_{31}}{n_{11}+n_{12}+n_{13}+y_{1}} &
\frac{n_{32}}{n_{21}+n_{22}+n_{23}+y_{2}} &
\frac{n_{33}}{n_{31}+n_{32}+n_{33}+y_{3}}
\end{array}\right) \ .
\]
%
It is important to realise that for an actual economic system the 
input--output matrix $\mathbf{P}$ can be determined only once 
\emph{the reference period of time chosen has come to an end}.

\medskip
\noindent
The utility of Leontief's stationary input--output matrix model is 
in its application for the purpose of {\bf forecasting}. This is 
done on the basis of an {\bf extrapolation}, namely by 
\emph{assuming} that an input--output matrix 
$\mathbf{P}_{\text{reference\ period}}$ obtained from data taken 
during a specific reference period also is valid (to an 
acceptable degree of accuracy) during a subsequent period, i.e.,
%
\be
\fbox{$\displaystyle
\mathbf{P}_{\text{subsequent\ period}}
\approx
\mathbf{P}_{\text{reference period}} \ ,
$}
\ee
%
or, in component form,
%
\be
\left.P_{ij}\right|_{\text{subsequent\ period}}
= \left.\frac{n_{ij}}{q_{j}}\right|_{\text{subsequent\ period}}
\approx
\left.P_{ij}\right|_{\text{reference\ period}}
= \left.\frac{n_{ij}}{q_{j}}\right|_{\text{reference\ period}} \ .
\ee
%
In this way it becomes possible to compute for a given (idealised) 
economic system approximate numbers of {\bf INPUT quantities} 
required during a near future production period from the known 
numbers of {\bf OUTPUT quantities} of the most recent production 
period. Long-term empirical experience has shown that this method 
generally leads to useful results to a reasonable approximation. 
All of these  calculations are grounded on linear relationships 
describing the quantitative aspects of stationary flows of goods, 
as we will soon elucidate.

%------------------------------------------------------------------
\subsection{Resource consumption matrix}
%------------------------------------------------------------------
The second matrix-valued quantity central to Leontief's stationary 
model is the {\bf resource consumption matrix} $\mathbf{R}$. This 
may be interpreted as providing a recipe for the amounts of the 
$m$~different kinds of external resources (the exogenous {\bf 
INPUT quantities}) that are needed in the production of the 
$n$~goods (the {\bf OUTPUT quantities}). Its elements are defined 
as the ratios
%
\be
\fbox{$\displaystyle
R_{ij} := \text{amounts\ (in\ units)\ required\ of\ resource\ $i$
\ for\ the\ production\ of\ one\ unit\ of\ good\ $j$} \ ,
$}
\ee
%
with $i=1,\ldots,m$ und $j=1,\ldots,n$. The rows of matrix 
$\mathbf{R}$ thus contain information concerning the 
$m$~resources, the columns information concerning the $n$~goods. 
Note that in general the $(m \times n)$ {\bf resource consumption 
matrix}~$\mathbf{R}$ is \emph{not (!)} a quadratic matrix and, 
therefore, in general \emph{not} invertible.

%%%%%%%%%%%%%%%%%%%%%%%%%%%%%%%%%%%%%%%%%%%%%%%%%%%%%%%%%%%%%%%%%%%
\section[Stationary linear flows of goods]%
{Stationary linear flows of goods}
\lb{sec:qstroeme}
%%%%%%%%%%%%%%%%%%%%%%%%%%%%%%%%%%%%%%%%%%%%%%%%%%%%%%%%%%%%%%%%%%%
%------------------------------------------------------------------
\subsection{Flows of goods: endogenous INPUT to total OUTPUT}
%------------------------------------------------------------------
We now turn to a quantitative description of the stationary {\bf 
flows of goods} that are associated with the {\bf total 
output}~$\vec{q}$ during a specific period of time considered. 
According to Leontief's Assumption~1, there exists a \emph{linear} 
functional relationship between the endogenous vector-valued {\bf 
INPUT quantity} $\vec{q}-\vec{y}$ and the vector-valued {\bf 
OUTPUT quantity}~$\vec{q}$. This may be represented in terms of a 
matrix-valued relationship as
%
\be
\lb{strom1}
\fbox{$\displaystyle
\vec{q}-\vec{y} = \mathbf{P}\vec{q}
\quad \Leftrightarrow \quad
q_{i}-y_{i} = \sum_{j=1}^{n}P_{ij}q_{j} \ ,
$}
\ee
%
with $i = 1, \ldots, n$, in which the {\bf input--output 
matrix}~$\mathbf{P}$ takes the role of mediating a mapping between 
either of these vector-valued quantities. According to 
Assumption~2, the elements of the {\bf input--output 
matrix}~$\mathbf{P}$ remain \emph{constant} for the period of time 
considered, i.e. the corresponding flows of goods are assumed to 
be {\bf stationary}.

\medskip
\noindent
Relation~(\ref{strom1}) may also be motivated from an alternative 
perspective that takes the {\bf physical sciences} as a guidline. 
Namely, the total numbers~$\vec{q}$ of the $n$~goods produced 
during the period of time considered which, by Assumption~3, are 
equal to the numbers supplied of the $n$~goods satisfy a {\bf  
conservation law}: ``whatever has been produced of the $n$ goods 
during the period of time considered \emph{cannot} get lost in 
this period.'' In quantitative terms this simple relationship may 
be cast into the form
%
\[
\underbrace{\vec{q}}_{\text{total output}}
=\underbrace{\vec{y}}_{\text{final demand (exogenous)}}
+\underbrace{\mathbf{P}\vec{q}}_{\text{deliveries to production sector (endogenous)}} \ .
\]
%

\medskip
\noindent
For computational purposes this central stationary flow of goods 
relation~(\ref{strom1}) may be rearranged as is convenient. In 
this context it is helpful to make use of the matrix identity  
$\vec{q}=\mathbf{1}\vec{q}$, where $\mathbf{1}$ denotes the
{\bf $\boldsymbol{(n \times n)}$-unit matrix} [cf.\ 
Eq.~(\ref{einmatr})].

%\pagebreak
\medskip
\noindent
{\bf Examples:}
%
\begin{itemize}
\item[(i)]~given/known: $\mathbf{P}$, $\vec{q}$

\medskip
\noindent
Then it applies that
%
\be
\lb{strom12}
\vec{y} = (\mathbf{1}-\mathbf{P})\vec{q}
\quad \Leftrightarrow \quad
y_{i} = \sum_{j=1}^{n}(\delta_{ij}-P_{ij})q_{j} \ ,
\ee
%
with $i = 1, \ldots, n$; $(\mathbf{1}-\mathbf{P})$ represents the 
invertible  {\bf technology matrix} of the economic system 
regarded.

\item[(ii)]~given/known: $\mathbf{P}$, $\vec{y}$

\medskip
\noindent
Then it holds that
%
\be
\lb{strom13}
\vec{q} = (\mathbf{1}-\mathbf{P})^{-1}\vec{y}
\quad \Leftrightarrow \quad
q_{i} = \sum_{j=1}^{n}(\delta_{ij}-P_{ij})^{-1}y_{j} \ ,
\ee
%
with $i = 1, \ldots, n$; $(\mathbf{1}-\mathbf{P})^{-1}$ here 
denotes the {\bf total demand matrix}, i.e., the inverse of the 
technology matrix.
\end{itemize}
%

%------------------------------------------------------------------
\subsection{Flows of goods: exogenous INPUT to total OUTPUT}
%------------------------------------------------------------------
Likewise, by Assumption~1, a \emph{linear} functional relationship 
is supposed to exist between the exogenous vector-valued {\bf 
INPUT quantity}~$\vec{v}$ and the vector-valued {\bf OUTPUT 
quantity}~$\vec{q}$. In matrix language this can be expressed by

%
\be
\lb{strom2}
\fbox{$\displaystyle
\vec{v} = \mathbf{R}\vec{q}
\quad\Leftrightarrow\quad
v_{i} = \sum_{j=1}^{n}R_{ij}q_{j} \ ,
$}
\ee
%
with $i = 1, \ldots, m$. By Assumption~2, the elements of the {\bf 
resource consumption matrix}~$\mathbf{R}$ remain \emph{constant} 
during the period of time considered, i.e., the corresponding 
resource flows are supposed to be {\bf stationary}.

\medskip
\noindent
By combination of Eqs.~(\ref{strom2}) and~(\ref{strom13}), it is 
possible to compute the numbers~$\vec{v}$ of resources required 
(during the period of time considered) for the production of the 
$n$~goods for given final demand~$\vec{y}$. It applies that
%
\be
\lb{strom22}
\vec{v} = \mathbf{R}\vec{q}
= \mathbf{R}(\mathbf{1}-\mathbf{P})^{-1}\vec{y}
\quad\Leftrightarrow\quad
v_{i} = \sum_{j=1}^{n}\sum_{k=1}^{n}R_{ij}
(\delta_{jk}-P_{jk})^{-1}y_{k} \ ,
\ee
%
with $i = 1, \ldots, m$.

\medskip
\noindent
\underline{\bf GDC:} For problems with $n \leq 5$, and known matrices $\mathbf{P}$ and $\mathbf{R}$, Eqs.~(\ref{strom12}), (\ref{strom13}) and (\ref{strom22}) can be immediately used to calculate the quantities $\vec{q}$ from given quantities $\vec{y}$, or vice versa.

%%%%%%%%%%%%%%%%%%%%%%%%%%%%%%%%%%%%%%%%%%%%%%%%%%%%%%%%%%%%%%%%%%%
\section[Outlook]%
{Outlook}
\lb{sec:geldstroeme}
%%%%%%%%%%%%%%%%%%%%%%%%%%%%%%%%%%%%%%%%%%%%%%%%%%%%%%%%%%%%%%%%%%%
Leontief's input--output matrix model may be extended in a 
straightforward fashion to include more advanced considerations of 
{\bf economic theory}. Supposing a closed though not necessarily 
stationary economic system~$G$ comprising $n$ interdependent {\bf 
economic agents} producing $n$ different goods, one may assign 
{\bf monetary values} to the {\bf INPUT quantity}~$\vec{v}$ as 
well as to the {\bf OUTPUT quantities} $\vec{q}$ and $\vec{y}$ of 
the system. Besides the numbers of goods produced and the 
associated flows of goods one may monitor with respect to~$G$ for 
a given period of time, one can in addition analyse in time and 
space the {\bf amount of money} coupled to the different goods, 
and the corresponding {\bf flows of money}. However, contrary to 
the number of goods, in general there does \emph{not} exist a 
{\bf conservation law} for the amount of money with respect 
to~$G$. This may render the analysis of flows of money more 
difficult, because, in the sense of an {\bf increase in value}, 
\emph{money can either be generated inside~$G$ during the period 
of time considered or it can likewise be annihilated}; it is 
\emph{not} just limited to either flowing into respectively 
flowing out of $G$. Central to considerations of this kind is a 
{\bf balance equation} for the amount of money contained in $G$ 
during a given period of time, which is an \emph{additive} 
quantity. Such balance equations constitute familiar tools in {\bf 
Physics} (cf. Herrmann (2003)~\ct[p~7ff]{her2003}). Its structure 
in the present case is given by\footnote{Here the symbols CU and 
TU denote ``currency units'' and ``time units,'' respectively.} 
%
\[
\left(\begin{array}{c}
\text{{\bf rate\ of\ change\ in\ time}} \\
\text{of\ the\ {\bf amount of money}}\\
\text{in}\ G\ \text{[in CU/TU]}
\end{array}\right)
= \left(\begin{array}{c}
\text{{\bf flux of money}} \\
\text{into}\ G\ \text{[in CU/TU]}
\end{array}\right)
+ \left(\begin{array}{c}
\text{{\bf rate of generation of money}} \\
\text{in}\ G\ \text{[in CU/TU]}
\end{array}\right) \ .
\]
%
Note that, with respect to $G$, both fluxes of money and rates of 
generation of money can in principle possess either sign, positive 
or negative. To deal with these quantitative issues properly, one 
requires the technical tools of the {\bf differential and integral 
calculus} which we will discuss at an elementary level in 
Chs.~\ref{ch7} and~\ref{ch8}. We make contact here with the 
interdisciplinary science of {\bf Econophysics} (cf., e.g., 
Bouchaud and Potters (2003) \ct{boupot2003}), a very interesting 
and challenging subject which, however, is beyond the scope of 
these lecture notes.


\medskip
\noindent
Leontief's input--output matrix model, and its possible extension 
as outlined here, provide the quantitative basis for 
considerations of economical ratios of the kind
%
\[
\frac{\text{OUTPUT [in~$\text{units}$]}}{\text{INPUT 
[in~$\text{units}$]}} \ ,
\]
%
as mentioned in the Introduction. In addition, \emph{dimensionless}
(scale-invariant) ratios of the form
%
\[
\frac{\text{REVENUE [in~$\text{CU}$]}}{\text{COSTS 
[in~$\text{CU}$]}} \ ,
\]
%
%
referred to as {\bf economic efficiency}, can be computed for and 
compared between different economic systems and their underlying 
production sectors. In Ch.~\ref{ch7} we will briefly reconsider 
this issue.

%%%%%%%%%%%%%%%%%%%%%%%%%%%%%%%%%%%%%%%%%%%%%%%%%%%%%%%%%%%%%%%%%%%
%%%%%%%%%%%%%%%%%%%%%%%%%%%%%%%%%%%%%%%%%%%%%%%%%%%%%%%%%%%%%%%%%%%
