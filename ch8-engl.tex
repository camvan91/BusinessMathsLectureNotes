%%%%%%%%%%%%%%%%%%%%%%%%%%%%%%%%%%%%%%%%%%%%%%%%%%%%%%%%%%%%%%%%%%%
%  File name: ch8-engl.tex
%  Title:
%  Version: 11.09.2015 (hve)
%%%%%%%%%%%%%%%%%%%%%%%%%%%%%%%%%%%%%%%%%%%%%%%%%%%%%%%%%%%%%%%%%%%
%%%%%%%%%%%%%%%%%%%%%%%%%%%%%%%%%%%%%%%%%%%%%%%%%%%%%%%%%%%%%%%%%%%
\chapter[Integral calculus of real-valued functions]%
{Integral calculus of real-valued functions of one real variable}
\lb{ch8}
%%%%%%%%%%%%%%%%%%%%%%%%%%%%%%%%%%%%%%%%%%%%%%%%%%%%%%%%%%%%%%%%%%%
%\hfill\hbox{\fbox{\vbox{\hsize=10cm
%Der Inhalt dieses Kapitels dient unter anderem der Vorbereitung
%auf elementare Teilaspekte der Wahrscheinlichkeitstheorie, welche
%im Rahmen der Module 0.1.3 WISS: Einf\"uhrung in das
%wissenschaftliche Arbeiten und die empirische Sozialforschung
%und 1.3.2 MARE: Marketing Research in der Schlie\ss enden 
%Statistik ben\"otigt wird.
%}}}

\vspace{10mm}
\noindent
In the final chapter of these lecture notes we give a brief 
overview of the main definitions and laws of the {\bf integral 
calculus} of real-valued functions of one variable. Subsequently 
we consider a simple application of this tool in {\bf economic 
theory}.

%%%%%%%%%%%%%%%%%%%%%%%%%%%%%%%%%%%%%%%%%%%%%%%%%%%%%%%%%%%%%%%%%%%
\section[Indefinite integrals]{Indefinite integrals}
\lb{sec:unbint}
%%%%%%%%%%%%%%%%%%%%%%%%%%%%%%%%%%%%%%%%%%%%%%%%%%%%%%%%%%%%%%%%%%%
\underline{\bf Def.:} Let $f$ be a continuous real-valued function 
of one real variable and $F$ a differentiable real-valued function 
of the same real variable, with $D(f)=D(F)$. Given that $f$ and 
$F$ are related according to
%
\be
\fbox{$\displaystyle
F^{\prime}(x) = f(x)
\quad\quad\text{\emph{for all}}\quad x\in D(f) \ ,
$}
\ee
%
then $F$ is referred to as a {\bf primitive} of $f$.

\medskip
\noindent
\underline{\bf Remark:} The primitive of a given continuous 
real-valued function $f$ \emph{cannot} be unique. By the rules of 
differentiation discussed in Sec.~\ref{sec:ablt}, besides $F$ 
also $F+c$, with $c \in \mathbb{R}$ a real-valued constant, 
constitutes a primitive of $f$ since $(c)^{\prime} = 0$.

\medskip
\noindent
\underline{\bf Def.:} If $F$ is a primitive of a continuous 
real-valued function $f$ of one real variable, then
%
\be
\fbox{$\displaystyle
\int f(x)\,{\rm d}x = F(x) + c \ , \quad
c~=~\text{constant}~\in~\mathbb{R} \ ,
\quad \text{with}\ F^{\prime}(x) = f(x)
$}
\ee
%
defines the {\bf indefinite integral} of the function $f$. The 
following names are used to refer to the different ingredients in 
this expression:
%
\begin{itemize}
\item $x$ --- the {\bf integration variable},
\item $f(x)$ --- the {\bf integrand},
\item ${\rm d}x$ --- the {\bf differential}, and, lastly,
\item $c$ --- the {\bf constant of integration}.
\end{itemize}
%

\medskip
\noindent
For the elementary, continuous real-valued functions of one 
variable introduced in Sec.~\ref{sec:fkt}, the following rules of 
indefinite integration apply:

\medskip
\noindent
{\bf Rules of indefinite integration}
%
\begin{enumerate}
\item $\int \alpha\,{\rm d}x = \alpha x + c$ with $\alpha
= \text{constant} \in \mathbb{R}$ \hfill ({\bf constants})
\item $\displaystyle \int x\,{\rm d}x = \frac{x^{2}}{2} + c$
 \hfill ({\bf linear functions})
\item $\displaystyle \int x^{n}\,{\rm d}x = \frac{x^{n+1}}{n+1} + 
c$ for $n \in \mathbb{N}$ \hfill ({\bf natural power-law 
functions})
\item $\displaystyle \int x^{\alpha}\,{\rm d}x = 
\frac{x^{\alpha+1}}{\alpha+1}
+ c$ for $\alpha \in \mathbb{R}\backslash\{-1\}$ and
$x \in \mathbb{R}_{> 0}$ \hfill ({\bf general power-law functions})
\item $\displaystyle \int a^{x}\,{\rm d}x = \frac{a^{x}}{\ln(a)} + c$ for $a \in \mathbb{R}_{> 0}\backslash\{1\}$
\hfill ({\bf exponential functions})
\item $\displaystyle \int e^{ax}\,{\rm d}x = \frac{e^{ax}}{a} + c$
for $a \in \mathbb{R}\backslash\{0\}$
\hfill ({\bf natural exponential functions})
\item $\int x^{-1}\,{\rm d}x = \ln|x| + c$ for $x \in \mathbb{R}\backslash\{0\}$.
\end{enumerate}
%
Special methods of integration need to be employed when the 
integrand consists of a concatanation of elementary real-valued 
functions. Here we provide a list with the main tools for this 
purpose. For differentiable real-valued functions $f$ and $g$, it 
holds that
%
\begin{enumerate}
\item $\int(\alpha f(x) \pm \beta g(x))\,{\rm d}x
= \alpha\int f(x)\,{\rm d}x \pm \beta\int g(x)\,{\rm d}x$ \\
with $\alpha,\beta = \text{constant} \in \mathbb{R}$
\hfill ({\bf summation rule})
\item $\int f(x)g^{\prime}(x)\,{\rm d}x = f(x)g(x)
- \int f^{\prime}(x)g(x)\,{\rm d}x$
\hfill ({\bf integration by parts})
\item $\int f(g(x))g^{\prime}(x)\,{\rm d}x
\overbrace{=}^{u=g(x)\ \text{and}\ {\rm d}u=g^{\prime}(x){\rm d}x}
\int f(u)\,{\rm d}u = F(g(x)) + c$
\hfill ({\bf substitution method})
\item $\displaystyle \int\frac{f^{\prime}(x)}{f(x)}\,{\rm d}x
= \ln|f(x)| + c$ for $f(x) \neq 0$\hfill ({\bf logarithmic integration}).
\end{enumerate}
%

%%%%%%%%%%%%%%%%%%%%%%%%%%%%%%%%%%%%%%%%%%%%%%%%%%%%%%%%%%%%%%%%%%%
\section[Definite integrals]{Definite integrals}
\lb{sec:bint}
%%%%%%%%%%%%%%%%%%%%%%%%%%%%%%%%%%%%%%%%%%%%%%%%%%%%%%%%%%%%%%%%%%%
\medskip
\noindent
\underline{\bf Def.:} Let $f$ be a real-valued function of one variable which is continuous on an interval $[a,b] \subset D(f)$, and let $F$ be a primitive of $f$. Then the expression
%
\be
\fbox{$\displaystyle
\int_{a}^{b}f(x)\,{\rm d}x
= {\displaystyle\left.F(x)\right|_{x=a}^{x=b}}
= F(b) - F(a)
$}
\ee
%
defines the {\bf definite integral} of $f$ in the 
{\bf limits of integration} $a$ and $b$.

\medskip
\noindent
For definite integrals the following general rules apply:
%
\begin{enumerate}
\item $\displaystyle\int_{a}^{a}f(x)\,{\rm d}x = 0$
\hfill ({\bf identical limits of integration})
\item $\displaystyle\int_{b}^{a}f(x)\,{\rm d}x = 
-\int_{a}^{b}f(x)\,{\rm d}x$
\hfill ({\bf interchange of limits of integration})
\item $\displaystyle\int_{a}^{b}f(x)\,{\rm d}x = 
\int_{a}^{c}f(x)\,{\rm d}x
+ \int_{c}^{b}f(x)\,{\rm d}x$ for $c \in [a,b]$
\hfill ({\bf split of integration interval}).
\end{enumerate}
%

\medskip
\noindent
\underline{\bf Remark:} The main qualitative difference between an 
(i)~indefinite integral and a (ii)~definite integral of a 
continuous real-valued function of one variable reveals intself in 
the different kinds of outcome: while (i)~yields as a result a 
real-valued (primitive) \emph{function}, (ii)~simply yields a 
single real \emph{number}.

\medskip
\noindent
\underline{\bf GDC:} For a stored real-valued function, the 
evaluation of a definite integral can be performed in mode {\tt 
CALC} with the pre-programmed function $\int${\tt f(x)dx}. The 
corresponding limits of integration need to be specified 
interactively.

\medskip
\noindent
As indicated in Sec.~\ref{sec:elast}, the scale-invariant 
power-law functions $f(x)=x^{\alpha}$ for $\alpha \in \mathbb{R}$
and $x \in \mathbb{R}_{> 0}$ play a special role in practical 
applications. For $x \in \left[a,b\right] \subset \mathbb{R}_{> 
0}$ and $\alpha \neq -1$ it holds that
%
\be
\int_{a}^{b}x^{\alpha}\,{\rm d}x
= \left.\frac{x^{\alpha+1}}{\alpha+1}\right|_{x=a}^{x=b}
= \frac{1}{\alpha+1}\left(b^{\alpha+1}-a^{\alpha+1}\right) \ .
\ee
%
Problematic in this context can be considerations of taking limits 
of the form $a \to 0$ resp. $b \to \infty$, since for either of 
the two cases
%
\begin{itemize}
\item[(i)] case $\alpha < -1$:
%
\be
\lim_{a \to 0}\int_{a}^{b}x^{\alpha}\,{\rm d}x \to \infty \ ,
\ee
%
\item[(ii)] case $\alpha > -1$:
%
\be
\lim_{b \to \infty}\int_{a}^{b}x^{\alpha}\,{\rm d}x \to \infty \ ,
\ee
%
\end{itemize}
%
one ends up with {\bf divergent} mathematical expressions.

%%%%%%%%%%%%%%%%%%%%%%%%%%%%%%%%%%%%%%%%%%%%%%%%%%%%%%%%%%%%%%%%%%%
\section[Applications in economic theory]{Applications in economic 
theory}
\lb{sec:intanw}
%%%%%%%%%%%%%%%%%%%%%%%%%%%%%%%%%%%%%%%%%%%%%%%%%%%%%%%%%%%%%%%%%%%
The starting point shall be a simple market situation for a single 
product. For this product, on the one-hand side, there be a {\bf 
demand function}~$N(p)$ (in $\text{units}$) which is monotonously 
decreasing on the price interval $[p_{u},p_{o}]$; the limit values 
$p_{u}$ and $p_{o}$ denote the minimum and maximum prices per unit 
(in CU/u) acceptable for the product. On the other hand, the 
market situation be described by a {\bf supply function}~$A(p)$ 
(in $\text{units}$) which is monotonously increasing on 
$[p_{u},p_{o}]$.

\medskip
\noindent
The {\bf equilibrium unit price} $p_{M}$ (in 
$\text{CU}/\text{unit}$) for this product is defined by assuming a 
state of {\bf economic equilibrium} of the market, quantitatively 
expressed by the condition
%
\be
A(p_{M}) = N(p_{M}) \ .
\ee
%
Geometrically, this condition defines common points of 
intersection for the functions $A(p)$ and $N(p)$ (when they exist).

\medskip
\noindent
\underline{\bf GDC:} Common points of intersection for stored 
functions $f$ and $g$ can be easily determined interactively in 
mode {\tt CALC} employing the routine {\tt intersect}.

\medskip
\noindent
In a drastically simplified fashion, we now turn to compute the 
revenue made on the market by the suppliers of a new product for 
either of three possible {\bf strategies of market entry}.
%
\begin{enumerate}

\item {\bf Strategy~1:} The revenue obtained by the suppliers when 
the new product is being sold straight at the equilibrium unit 
price $p_{M}$, in an amount $N(p_{M})$, is simply given by
%
\be
U_{1} = U(p_{M}) = p_{M}N(p_{M}) \qquad
(\text{in\ CU}) \ .
\ee
%

\item {\bf Strategy~2:} Some consumers would be willing to 
purchase the product intially also at a unit price which is higher 
than $p_{M}$. If, hence, the suppliers decide to offer the product 
initially at a unit price $p_{o} > p_{M}$, and then, in order to 
generate further demand, to continuously\footnote{This is a strong 
mathematical assumption aimed at facilitating the actual 
calculation to follow.} (!) \emph{reduce} the unit price to the 
lower $p_{M}$, the revenue obtained yields the larger value
%
\be
U_{2} = U(p_{M}) + \int_{p_{M}}^{p_{o}}N(p)\,{\rm d}p \ .
\ee
%
Since the amount of money
%
\be
K := \int_{p_{M}}^{p_{o}}N(p)\,{\rm d}p \qquad
(\text{in\ CU})
\ee
%
is (theoretically) safed by the consumers when the product is 
introduced to the market according to strategy~1, this amount is 
referred to in the economic literature as {\bf consumer surplus}.

\item {\bf Strategy~3:} Some suppliers would be willing to 
introduce the product to the market initially at a unit price 
which is lower than $p_{M}$. If, hence, the suppliers decide to 
offer the product initially at a unit price $p_{u} < p_{M}$, and 
then to continuously\footnote{See previous footnote.} (!) 
\emph{raise} it to the higher $p_{M}$, the revenue obtaines 
amounts to the smaller value
%
\be
U_{3} = U(p_{M}) - \int_{p_{u}}^{p_{M}}A(p)\,{\rm d}p \ .
\ee
%
Since the suppliers (theoretically) earn the extra amount
%
\be
P := \int_{p_{u}}^{p_{M}}A(p)\,{\rm d}p \qquad
(\text{in\ CU})
\ee
%
when the product is introduced to the market according to 
strategy~1, this amount is referred to in the economic literature 
as {\bf producer surplus}.

\end{enumerate}
%

%%%%%%%%%%%%%%%%%%%%%%%%%%%%%%%%%%%%%%%%%%%%%%%%%%%%%%%%%%%%%%%%%%%
%%%%%%%%%%%%%%%%%%%%%%%%%%%%%%%%%%%%%%%%%%%%%%%%%%%%%%%%%%%%%%%%%%%