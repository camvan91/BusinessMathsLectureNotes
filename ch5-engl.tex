%%%%%%%%%%%%%%%%%%%%%%%%%%%%%%%%%%%%%%%%%%%%%%%%%%%%%%%%%%%%%%%%%%%
%  File name: ch5-engl.tex
%  Title:
%  Version: 11.09.2014 (hve)
%%%%%%%%%%%%%%%%%%%%%%%%%%%%%%%%%%%%%%%%%%%%%%%%%%%%%%%%%%%%%%%%%%%
%%%%%%%%%%%%%%%%%%%%%%%%%%%%%%%%%%%%%%%%%%%%%%%%%%%%%%%%%%%%%%%%%%%
\chapter[Linear programming]{Linear programming}
\lb{ch5}
%%%%%%%%%%%%%%%%%%%%%%%%%%%%%%%%%%%%%%%%%%%%%%%%%%%%%%%%%%%%%%%%%%%
%\hfill\hbox{\fbox{\vbox{\hsize=10cm
%Der Inhalt dieses Kapitels ist relevant f\"ur das
%Verst\"andnis der in Kapitel 5.3 und 5.4 des Lehrbuchs von
%Schrey\"ogg und Koch (2010)~\ct{schkoc2010} behandelten
%quantitativen Themen.
%%Gleichfalls f\"ur Kapitel 5.3
%%des Lehrbuchs von Schmalen und Pechtl 2006~\ct{schpec2006}.
%}}}

\vspace{10mm}
\noindent
On the backdrop of the {\bf economic principle}, we discuss in 
this chapter a special class of quantitative problems that 
frequently arise in specific practical applications in {\bf 
Business} and {\bf Management}. Generally one distinguishes 
between two variants of the {\bf economic principle}: either 
(i)~to draw maximum utility from limited resources, or (ii)~to 
reach a specific target with minimum effort (costs). With regard 
to the ratio $(\text{OUTPUT})/(\text{INPUT})$ put into focus in 
the Introduction, the issue is to find an {\bf optimal value} for 
this ratio under given {\bf boundary conditions}. This aim can be 
realised either (i)~by increasing the (positive) value of the 
numerator for fixed (positive) value of the denominator, or 
(ii)~by decreasing the (positive) value of the denominator for 
fixed (positive) value of the numerator. The class of quantitative 
problems to be looked at in some detail in this chapter typically 
relate to boundary conditions according to case~(i).

%%%%%%%%%%%%%%%%%%%%%%%%%%%%%%%%%%%%%%%%%%%%%%%%%%%%%%%%%%%%%%%%%%%
\section[Exposition of a quantitative problem]{Exposition of a
quantitative problem}
\lb{sec:lopeinf}
%%%%%%%%%%%%%%%%%%%%%%%%%%%%%%%%%%%%%%%%%%%%%%%%%%%%%%%%%%%%%%%%%%%
To be maximised is a (non-negative) real-valued quantity~$z$, 
which depends in a \emph{linear functional fashion} on a fixed 
number of ~$n$ (non-negative) real-valued variables $x_{1}, 
\ldots, x_{n}$. We suppose that the $n$~variables $x_{1}, \ldots, 
x_{n}$ in turn are constrained by a fixed number~$m$ of algebraic 
conditions, which also are assumed to depend on $x_{1}, \ldots, 
x_{n}$ in a \emph{linear fashion}. These $m$~constraints, or 
restrictions, shall have the character of imposing upper limits on 
$m$~different kinds of resources.

\medskip
\noindent
\underline{\bf Def.:}
Consider a matrix~$\mathbf{A} \in
\mathbb{R}^{m \times n}$, a vector~$\vec{b} \in
\mathbb{R}^{m \times 1}$, two vectors $\vec{c},\vec{x} \in
\mathbb{R}^{n \times 1}$, and a constant $d \in \mathbb{R}$. A quantitative problem of the form
%
\be
\fbox{$\displaystyle
\text{max}\left\{z=\vec{c}^{T}\cdot\vec{x}
+d\left|\mathbf{A}\vec{x}
\leq \vec{b}, \vec{x} \geq \vec{0}\right.\right\} \ ,
$}
\ee
%
or, expressed in terms of a component notation,
%
\begin{eqnarray}
\lb{lops1}
\text{max}\ z(x_{1},\ldots,x_{n})
= c_{1}x_{1} + \ldots + c_{n}x_{n} + d & & \\
a_{11}x_{1}+\ldots+a_{1n}x_{n} & \leq & b_{1} \\
%
 & \vdots & \nonumber \\
%
a_{m1}x_{1}+\ldots+a_{mn}x_{n} & \leq & b_{m} \\
%
x_{1} & \geq & 0 \\
%
 & \vdots & \nonumber \\
%
\lb{lopsm1}
x_{n} & \geq & 0 \ ,
\end{eqnarray}
%
is referred to as a  {\bf standard maximum problem of 
linear programming} with $n$~real-valued variables. The different 
quantities and relations appearing in this definition are 
called

\begin{itemize}

\item $z(x_{1},\ldots,x_{n})$ --- {\bf linear objective function}, the dependent variable,

\item $x_{1},\ldots,x_{n}$ --- $n$ {\bf independent variables},

\item $\mathbf{A}\vec{x} \leq \vec{b}$ --- $m$ {\bf restrictions},

\item $\vec{x} \geq \vec{0}$ ---
$n$ {\bf non-negativity constraints}.

\end{itemize}

\noindent
\underline{\bf Remark:} In an analogous fashion one may also formulate a {\bf standard minimum problem of linear programming}, which can be cast into the form
%
\[
\text{min}\left\{z=\vec{c}^{T}\cdot\vec{x}
+d\left|\mathbf{A}\vec{x}
\geq \vec{b}, \vec{x} \geq \vec{0}\right.\right\} \ .
\]
%
In this case, the components of the vector $\vec{b}$ need to be 
interpreted as lower limits on certain capacities.

\medskip
\noindent
For given linear objective function $z(x_{1},\ldots,x_{n})$, the set of points $\vec{x} = (x_{1},\ldots,x_{n})^{T}$ satisfying the condition
%
\be
\fbox{$\displaystyle
z(x_{1},\ldots,x_{n}) = C = \text{constant} \in \mathbb{R} \ ,
$}
\ee
%
for fixed value of $C$, is referred to as an {\bf isoquant} of 
$z$. {\bf Isoquants} of linear objective functions of $n=2$ 
independent variables constitute straight lines, of $n=3$ 
independent variables Euclidian planes, of $n=4$ independent 
variables Euclidian 3-spaces (or hyperplanes), and of $n\geq 5$ 
independent variables Euclidian $(n-1)$-spaces (or hyperplanes).

\medskip
\noindent
In the simplest cases of {\bf linear programming}, the linear {\bf 
objective function}~$z$ depends on $n=2$ {\bf variables}~$x_{1}$ 
and $x_{2}$ only. An illustrative and efficient method of solving 
problems of this kind will be looked at in the following section.

%%%%%%%%%%%%%%%%%%%%%%%%%%%%%%%%%%%%%%%%%%%%%%%%%%%%%%%%%%%%%%%%%%%
\section[Graphical solution method]%
{Graphical method for solving problems with two independent 
variables}
\lb{sec:lopgraf}
%%%%%%%%%%%%%%%%%%%%%%%%%%%%%%%%%%%%%%%%%%%%%%%%%%%%%%%%%%%%%%%%%%%
The systematic graphical solution method of standard maximum 
problems of {\bf linear programming} with $n=2$~independent 
variables comprises the following steps:
%
\begin{enumerate}

\item Derivation of the {\bf linear objective function}
%
$$
z(x_{1},x_{2})=c_{1}x_{1}+c_{2}x_{2}+d
$$
%
in dependence on the {\bf variables}~$x_{1}$ and $x_{2}$.

\item Identification in the $x_{1},x_{2}$--plane of the {\bf feasible region} $D$ of $z$ which is determined by the $m$ restrictions imposed on $x_{1}$ and $x_{2}$. Specifically, $D$ constitutes the domain of $z$ (cf. Ch.~\ref{ch7}).

\item Plotting in the $x_{1},x_{2}$--plane of the projection of the {\bf isoquant} of the linear objective function~$z$ which intersects the origin ($0=x_{1}=x_{2}$). When $c_{2} \neq 0$, this projection is described by the equation
%
\[
x_{2} = -(c_{1}/c_{2})x_{1} \ .
\]
%

\item Erecting in the origin of the $x_{1},x_{2}$--plane the {\bf direction of optimisation} for $z$ which is determined by the constant $z$-gradient
%
$$
(\boldsymbol{\nabla} z)^{T}
=\left(\begin{array}{c}\frac{\ptl z}{\ptl x_{1}} \\
\frac{\ptl z}{\ptl x_{2}}\end{array}\right)
=\left(\begin{array}{c} c_{1} \\ c_{2}\end{array}\right) \ .
$$
%

\item {\bf Parallel displacement} in the $x_{1},x_{2}$--plane of the projection of the $(0,0)$-isoquant of~$z$ along the direction of optimisation~$(\boldsymbol{\nabla} z)^{T}$ across the feasible region~$D$ out to a distance where the projected isoquant just about touches~$D$.

\item Determination of the {\bf optimal solution} $(x_{1{\rm O}},
x_{2{\rm O}})$ as the point resp.~set of points of intersection between the displaced projection of the $(0,0)$-isoquant of~$z$ and the \emph{far} boundary of~$D$.

\item Computation of the {\bf optimal value} of the linear objective function~$z_{{\rm O}}=z(x_{1{\rm O}},x_{2{\rm O}})$ from the optimal solution~$(x_{1{\rm O}}, x_{2{\rm O}})$.

\item Specification of potential {\bf remaining resources} by substitution of the optimal solution $(x_{1{\rm O}}, x_{2{\rm O}})$ into the $m$~restrictions.

\end{enumerate}
%
In general one finds that for a linear {\bf objective 
function}~$z$ with $n=2$ {\bf independent variables}~$x_{1}$ and 
$x_{2}$, the feasible region~$D$, when \emph{non-empty and 
bounded}, constitutes an area in the $x_{1},x_{2}$--plane with 
straight edges and a certain number of vertices. In these cases, 
the {\bf optimal values} of the linear objective function~$z$ are 
always to be found either at the vertices or on the edges of the 
feasible region~$D$. When $D$ is an empty set, then there exists 
no solution to the corresponding linear programming problem. When 
$D$ is unbounded, again there may not exist a solution to the 
linear programming problem, but this then depends on the specific 
circumstances that apply.

\medskip
\noindent
\underline{\bf Remark:} To solve a {\bf standard minimum problem 
of linear programming} with $n=2$ independent variables by means 
of the graphical method, one needs to parallelly displace in the 
$x_{1},x_{2}$--plane the projection of the $(0,0)$-isoquant of~$z$ 
along the direction of optimisation~$(\boldsymbol{\nabla} z)^{T}$ 
until contact is made with the feasible region~$D$ for the first 
time. The optimal solution is then given by the point resp.~set of 
points of intersection between the displaced projection of the 
$(0,0)$-isoquant of~$z$ and the \emph{near} boundary of~$D$.

%%%%%%%%%%%%%%%%%%%%%%%%%%%%%%%%%%%%%%%%%%%%%%%%%%%%%%%%%%%%%%%%%%%
\section[Dantzig's simplex algorithm]%
{Dantzig's simplex algorithm}
\lb{sec:lopsimplex}
%%%%%%%%%%%%%%%%%%%%%%%%%%%%%%%%%%%%%%%%%%%%%%%%%%%%%%%%%%%%%%%%%%%
The main disadvantage of the graphical solution method is its 
limitation to problems with only $n=2$~independent variables. In 
actual practice, however, one is often concerned with {\bf linear 
programming problems} that depend on \emph{more} than two {\bf 
independent variables}. To deal with these more complex problems 
in a systematic fashion, the US-American mathematician 
\href{http://www-groups.dcs.st-and.ac.uk/~history/Biographies/Dantzig_George.html}{George
Bernard Dantzig (1914--2005)} has devised during the 1940ies an 
efficient algorithm which can be programmed on a computer in a 
fairly straightforward fashion; cf. Dantzig 
(1949,1955)~\ct{dan1949,dan1955}.

\medskip
\noindent
In mathematics, {\bf simplex} is an alternative name used to refer 
to a convex polyhedron, i.e.,~a body of finite (hyper-)volume in 
two or more dimensions bounded by linear (hyper-)surfaces which 
intersect in linear edges and vertices. In general the feasible 
regions of linear programming problems constitute such simplexes. 
Since the {\bf optimal solutions} for the {\bf independent 
variables} of {\bf linear programming problems},
%corresponding to an extremal value of the objective function,
when they exist, are always to be found at a vertex or along an 
edge of simplex feasible regions, Dantzig developed his so-called 
{\bf simplex algorithm} such that it systematically scans the 
edges and vertices of a feasible region to identify the {\bf 
optimal solution} (when it exists) in as few steps as possible.

\medskip
\noindent
The starting point be a {\bf standard maximum problem of linear 
programming} with $n$~{\bf independent variables} in the form of 
relations~(\ref{lops1})--(\ref{lopsm1}). First, by introducing $m$ 
non-negative {\bf slack variables} $s_{1}, \ldots, s_{m}$, one 
transforms the $m$ linear {\bf restrictions} (inequalities) into 
an equivalent set of $m$ linear equations. In this way, potential 
differences between the left-hand and the right-hand sides of the 
$m$ inequalities are represented by the slack variables. In 
combination with the defining equation of the linear {\bf objective
function}~$z$, one thus is confronted with a system of $1+m$ 
linear algebraic equations for the $1+n+m$ variables $z, x_{1},
\ldots, x_{n}, s_{1}, \ldots, s_{m}$, given by

\medskip
\noindent
{\bf Maximum problem of linear programming in canonical form}
\nopagebreak
%
\begin{eqnarray}
\lb{lopk1}
z - c_{1}x_{1} - c_{2}x_{2} - \ldots - c_{n}x_{n} & = & d \\
%
a_{11}x_{1} + a_{12}x_{2} + \ldots + a_{1n}x_{n} + s_{1} & = & b_{1} \\
%
a_{21}x_{1} + a_{22}x_{2} + \ldots + a_{2n}x_{n} + s_{2} & = & b_{2} \\
%
 & \vdots & \nonumber \\
%
\lb{lopkm1}
a_{m1}x_{1} + a_{m2}x_{2} + \ldots + a_{mn}x_{n} + s_{m} & = & b_{m} \ .
\end{eqnarray}
%
As discussed previously in Ch.~\ref{ch3}, a system of linear 
algebraic equations of format $(1+m) \times (1+n+m)$ is 
\emph{under-determined} and so, at most, allows for \emph{multiple 
solutions}. The general $(1+n+m)$-dimensional solution vector
%
\be
\vec{x}_{L} = \left(z_{L}, x_{1, L}, \ldots, x_{n, L},
s_{1, L}, \ldots, s_{m, L}\right)^{T}
\ee
%
thus contains $n$ variables the values of which can be chosen 
\emph{arbitrarily}. It is very important to be aware of this fact. 
It implies that, given the linear system is solvable in the first 
place, one has a \emph{choice} amongst different solutions, and so 
one can pick the solution which proves {\bf optimal} for the given 
problem at hand. {\bf Dantzig's simplex algorithm} constitues a 
tool for determining such an {\bf optimal solution} in a 
systematic way.

\medskip
\noindent
Let us begin by transferring the coefficients and right-hand sides 
(RHS) of the under-determined linear system introduced above into 
a particular kind of {\bf simplex tableau}.

\medskip
\noindent
{\bf Initial simplex tableau}
%
\be
\lb{tableau1}
\begin{array}{c|rrcr|cccc|r}
z & x_{1} & x_{2} & \ldots & x_{n} & s_{1} & s_{2} & \ldots
& s_{m} & \text{RHS} \\
\hline
1 & -c_{1} & -c_{2} & \ldots & -c_{n} & 0 & 0 & \ldots & 0 & d \\
\hline
0 & a_{11} & a_{12} & \ldots & a_{1n} & 1 & 0 & \ldots & 0 & b_{1} \\
0 & a_{21} & a_{22} & \ldots & a_{2n} & 0 & 1 & \ldots & 0 & b_{2} \\
\vdots & \vdots & \vdots & \ddots & \vdots & \vdots & \vdots &
\ddots & \vdots & \vdots \\
0 & a_{m1} & a_{m2} & \ldots & a_{mn} & 0 & 0 & \ldots & 1 & b_{m}
\end{array}
\ee
%
In such a {\bf simplex tableau} one distinguishes so-called {\bf 
basis variables} from {\bf non-basis variables}. Basis variables 
are those that contain in their respective columns in the number 
tableau a $(1+m)$-component canonical unit vector [cf.\ 
Eq.~(\ref{kanbasis})]; in total the {\bf simplex tableau} contains 
$1+m$~of these. Non-basis variables are the remaining ones that do 
\emph{not} contain a canonical basis vector in their respective 
columns; there exist $n$ of this kind. The complete basis can thus 
be perceived as spanning a $(1+m)$-dimensional Euclidian space 
${\mathbb R}^{1+m}$. Initially, always $z$ and the $m$ slack 
variables $s_{1}, \ldots, s_{m}$ constitute the basis variables, 
while the $n$ independent variables $x_{1}, \ldots, x_{n}$ 
classify as non-basis variables [cf.\ the initial 
tableau~(\ref{tableau1})]. The corresponding so-called (first) 
{\bf basis solution} has the general appearance
%
\[
\vec{x}_{B_{1}} = \left(z_{B_{1}},
x_{1,B_{1}}, \ldots, x_{n, B_{1}},
s_{1, B_{1}}, \ldots, s_{m, B_{1}}\right)^{T}
= \left(d, 0, \ldots, 0,
b_{1}, \ldots, b_{m}\right)^{T} \ ,
\]
%
since, for simplicity, each of the $n$ arbitrarily specifiable 
non-basis variables may be assigned the special value zero. In 
this respect basis solutions will always be \emph{special 
solutions} (as opposed to general ones) of the under-determined 
system~(\ref{lopk1})--(\ref{lopkm1}) --- the maximum problem of 
linear programming in canonical form.

\medskip
\noindent
Central aim of the {\bf simplex algorithm} is to bring as many of 
the $n$~{\bf independent variables}~$x_{1}, \ldots, x_{n}$ as 
possible into the $(1+m)$-dimensional basis, at the expense of one 
of the $m$~{\bf slack variables}~$s_{1}, \ldots, s_{m}$, one at a 
time, in order to construct successively more favourable special 
vector-valued solutions to the optimisation problem at hand. 
Ultimately, the {\bf simplex algorithm} needs to be viewed as a 
special variant of Gau\ss ian elimination as discussed in 
Ch.~\ref{ch3}, with a set of systematic instructions concerning 
allowable equivalence transformations of the underlying 
under-determined linear system~(\ref{lopk1})--(\ref{lopkm1}), 
resp.\ the initial {\bf simplex tableau}~(\ref{tableau1}). This 
set of systematic algebraic simplex operations can be summarised 
as follows:

\medskip
\noindent
{\bf Simplex operations}
\nopagebreak
\begin{itemize}
\item[S1:] Does the current simplex tableau show $-c_{j} \geq 0$
for all $j \in \{1,\ldots,n\}$? If so, then the corresponding basis solution is {\bf optimal}. {\it END}. Otherwise goto S2.

\item[S2:] Choose a {\bf pivot column index} $j^{*} \in
\{1,\ldots,n\}$ such that $-c_{j^{*}}:= \text{min}\{-c_{j}|j
\in \{1,\ldots,n\}\} < 0$.

\item[S3:] Is there a row index $i^{*} \in \{1,\ldots,m\}$ such that $a_{i^{*}j^{*}} > 0$? If not, the objective function $z$ is unbounded from above. {\it END}. Otherwise goto S4.

\item[S4:] Choose a {\bf pivot row index} $i^{*}$ such that
$a_{i^{*}j^{*}} > 0$ and $b_{i^{*}}/a_{i^{*}j^{*}}
:=\text{min}\{b_{i}/a_{i^{*}j^{*}}|a_{i^{*}j^{*}} > 0,
i \in \{1,\ldots,m\}\}$. Perform a {\bf pivot operation} with the 
{\bf pivot element} $a_{i^{*}j^{*}}$. Goto S1.
\end{itemize}

\medskip
\noindent
When the final {\bf simplex tableau} has been arrived at, one 
again assigns the non-basis variables the value zero. The values 
of the final basis variables corresponding to the {\bf optimal 
solution} of the given {\bf linear programming problem} are then 
to be determined from the final {\bf simplex tableau} by backward 
substitution, beginning at the bottom row. Note that slack 
variables with positive values belonging to the basis variables in 
the {\bf optimal solution} provide immediate information on 
existing remaining capacities in the problem at hand.

%%%%%%%%%%%%%%%%%%%%%%%%%%%%%%%%%%%%%%%%%%%%%%%%%%%%%%%%%%%%%%%%%%%
%%%%%%%%%%%%%%%%%%%%%%%%%%%%%%%%%%%%%%%%%%%%%%%%%%%%%%%%%%%%%%%%%%%
