%%%%%%%%%%%%%%%%%%%%%%%%%%%%%%%%%%%%%%%%%%%%%%%%%%%%%%%%%%%%%%%%%%%
%  File name: intro-engl.tex
%  Version: 11.09.2015 (hve)
%%%%%%%%%%%%%%%%%%%%%%%%%%%%%%%%%%%%%%%%%%%%%%%%%%%%%%%%%%%%%%%%%%%
\addcontentsline{toc}{chapter}{Introduction}
%%%%%%%%%%%%%%%%%%%%%%%%%%%%%%%%%%%%%%%%%%%%%%%%%%%%%%%%%%%%%%%%%%%
\chapter*{Introduction}
%%%%%%%%%%%%%%%%%%%%%%%%%%%%%%%%%%%%%%%%%%%%%%%%%%%%%%%%%%%%%%%%%%%
These lecture notes contain the entire material of the 
quantitative methods part of the first semester module {\bf 0.1.1 
IMQM: Introduction to Management and its Quantitative Methods} at 
Karlshochschule International University. The aim is to provide a 
selection of tried-and-tested mathematical tools that proved 
efficient in actual practical problems of {\bf Economics} and {\bf 
Management}. These tools constitute the foundation for a 
systematic treatment of the typical kinds of quantitative problems 
one is confronted with in a Bachelor degree programme. 
Nevertheless, they provide a sufficient amount of points of 
contact with a quantitatively oriented subsequent Master degree 
programme in {\bf Economics}, {\bf Management}, or the {\bf Social 
Sciences}.

\medskip
\noindent
The prerequisites for a proper understanding of these lecture 
notes are modest, as they do not go much beyond the basic A-levels 
standards in {\bf Mathematics}. Besides the four fundamental 
arithmetical operations of addition, subtraction, multiplication 
and division of real numbers, you should be familiar, e.g., with 
manipulating fractions, dealing with powers of real numbers, the 
binomial formulae, determining the point of intersection for two 
straight lines in the Euclidian plane, solving a quadratic 
algebraic equation, and the rules of differentiation of 
real-valued functions of one variable.

\medskip
\noindent
It might be useful for the reader to have available a modern 
{\bf graphic display calculator (GDC)} for dealing with some of 
the calculations that necessarily arise along the way, when 
confronted with specific quantitative problems. Some current 
models used in public schools and in undergraduate studies are, 
amongst others,
%
\begin{itemize}
\item Texas Instruments \emph{TI--84 plus},
\item Casio \emph{CFX--9850GB PLUS}.
\end{itemize}
%
However, the reader is strongly encouraged to think about 
resorting, as an alternative, to a {\bf spreadsheet programme} 
such as EXCEL or OpenOffice to handle the calculations one 
encounters in one's quantitative work.

\medskip
\noindent
The central theme of these lecture notes is the acquisition and 
application of a number of effective mathematical methods in a 
business oriented environment. In particular, we hereby focus on 
{\bf quantitative processes} of the sort
%
\[
\text{INPUT} \rightarrow \text{OUTPUT} \ ,
\]
%
for which different kinds of {\bf functional relationships} 
between some numerical {\bf INPUT quantities} and some numerical 
{\bf OUTPUT quantities} are being considered. Of special interest 
in this context will be {\bf ratios} of the structure
%
\[
\frac{\text{OUTPUT}}{\text{INPUT}} \ .
\]
%
In this respect, it is a general objective in {\bf Economics} to 
look for ways to optimise the value of such ratios (in favour of 
some {\bf economic agent}), either by seeking to increase the 
OUTPUT when the INPUT is confined to be fixed, or by seeking to 
decrease the INPUT when the OUTPUT is confined to be fixed. 
Consequently, most of the subsequent considerations in these 
lecture notes will therefore deal with issues of {\bf 
optimisation} of given {\bf functional relationships} between some 
{\bf variables}, which manifest themselves either in {\bf 
minimisation} or in {\bf maximisation} procedures.

\medskip
\noindent
The structure of these lecture notes is the following. Part~I 
presents selected mathematical methods from {\bf Linear Algebra}, 
which are discussed in Chs.~\ref{ch1} to \ref{ch5}. Applications 
of these methods focus on the quantitative aspects of flows 
of goods in simple economic models, as well as on problems in 
linear programming. In Part~II, which is limited to 
Ch.~\ref{ch6}, we turn to discuss elementary aspects of {\bf 
Financial Mathematics}. Fundamental principles of {\bf Analysis}, 
comprising differential and integral calculus for real-valued 
functions of one real variable, and their application to 
quantitative economic problems, are reviewed in Part~III; this
extends across Chs.~\ref{ch7} and \ref{ch8}.

\medskip
\noindent
We emphasise the fact that there are \emph{no} explicit examples 
nor exercises included in these lecture notes. These are reserved 
exclusively for the lectures given throughout term time.

\medskip
\noindent
Recommended textbooks accompanying the lectures are the works by 
Asano (2013)~\ct{asa2013}, Dowling (2009)~\ct{dow2009}, Dow\-ling 
(1990)~\ct{dow1990}, Bauer \emph{et al} (2008)~\ct{bauetal2008},
Bosch (2003)~\ct{bos2003}, and H\"ulsmann \emph{et al} 
(2005)~\ct{hueetal2005}. Some standard references of {\bf Applied 
Mathematics} are, e.g., Bronstein \emph{et al} 
(2005)~\ct{broetal2005} and Arens \emph{et al} 
(2008)~\ct{areetal2008}. Should the reader feel inspired by the 
aesthetics, beauty, ellegance and efficiency of the mathematical 
methods presented, and, hence, would like to know more about their 
background and relevance, as well as being introduced to further 
mathematical techniques of interest, she/he is recommended to take 
a look at the brilliant books by Penrose (2004)~\ct{pen2004}, 
Singh (1997)~\ct{sin1997}, Gleick(1987)~\ct{gle1987} and Smith 
(2007)~\ct{smi2007}. Note that most of the textbooks and 
monographs mentioned in this Introduction are available from 
the library at Karls\-hochschule International University.

\medskip
\noindent
Finally, we draw the reader's attention to the fact that the
*.pdf version of these lecture notes contains interactive features 
such as fully hyperlinked references to original publications at 
the websites \href{http://dx.doi.org}{{\tt dx.doi.org}} and 
\href{http://www.jstor.org}{{\tt jstor.org}}, as well 
as active links to biographical information on scientists that 
have been influential in the historical development of {\bf 
Mathematics}, hosted by the websites 
\href{http://www-history.mcs.st-and.ac.uk/}{The MacTutor History 
of Mathematics archive ({\tt www-history.mcs.st-and.ac.uk})} and 
\href{http://en.wikipedia.org/wiki/Main_Page}{{\tt 
en.wikipedia.org}}.

%%%%%%%%%%%%%%%%%%%%%%%%%%%%%%%%%%%%%%%%%%%%%%%%%%%%%%%%%%%%%%%%%%%
%%%%%%%%%%%%%%%%%%%%%%%%%%%%%%%%%%%%%%%%%%%%%%%%%%%%%%%%%%%%%%%%%%%