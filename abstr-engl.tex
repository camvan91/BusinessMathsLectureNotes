%%%%%%%%%%%%%%%%%%%%%%%%%%%%%%%%%%%%%%%%%%%%%%%%%%%%%%%%%%%%%%%%%%%
%  File name: abstr-engl.tex
%  Title:
%  Version: 16.09.2015 (hve)
%%%%%%%%%%%%%%%%%%%%%%%%%%%%%%%%%%%%%%%%%%%%%%%%%%%%%%%%%%%%%%%%%%%
\addcontentsline{toc}{chapter}{Abstract}
%%%%%%%%%%%%%%%%%%%%%%%%%%%%%%%%%%%%%%%%%%%%%%%%%%%%%%%%%%%%%%%%%%%
\chapter*{}
\vspace{-8ex}
\section*{Abstract}
%%%%%%%%%%%%%%%%%%%%%%%%%%%%%%%%%%%%%%%%%%%%%%%%%%%%%%%%%%%%%%%%%%%
{\small These lecture notes provide a self-contained introduction 
to the mathematical methods required in a Bachelor degree 
programme in Business, Economics, or Management. In particular, 
the topics covered comprise real-valued vector and matrix algebra, 
systems of linear algebraic equations, Leontief's stationary 
input--output matrix model, linear programming, elementary 
financial mathematics, as well as differential and integral 
calculus of real-valued functions of one real variable. A special 
focus is set on applications in quantitative economical modelling.}

\vspace{10mm}
\noindent
\underline{Cite as:} 
\href{http://arxiv.org/abs/1509.04333}{arXiv:1509.04333v2
[q-fin.GN]}
\vfill

\medskip
\noindent
These lecture notes were typeset in \LaTeXe.

%%%%%%%%%%%%%%%%%%%%%%%%%%%%%%%%%%%%%%%%%%%%%%%%%%%%%%%%%%%%%%%%%%%
%%%%%%%%%%%%%%%%%%%%%%%%%%%%%%%%%%%%%%%%%%%%%%%%%%%%%%%%%%%%%%%%%%%