%%%%%%%%%%%%%%%%%%%%%%%%%%%%%%%%%%%%%%%%%%%%%%%%%%%%%%%%%%%%%%%%%%%
%  File name: ch6-engl.tex
%  Title:
%  Version: 23.11.2017 (hve)
%%%%%%%%%%%%%%%%%%%%%%%%%%%%%%%%%%%%%%%%%%%%%%%%%%%%%%%%%%%%%%%%%%%
%%%%%%%%%%%%%%%%%%%%%%%%%%%%%%%%%%%%%%%%%%%%%%%%%%%%%%%%%%%%%%%%%%%
\chapter[Elementary financial mathematics]%
{Elementary financial mathematics}
\lb{ch6}
%%%%%%%%%%%%%%%%%%%%%%%%%%%%%%%%%%%%%%%%%%%%%%%%%%%%%%%%%%%%%%%%%%%
%\hfill\hbox{\fbox{\vbox{\hsize=10cm
%Der Inhalt dieses Kapitels diskutiert grundlegende
%finanzmathematische Themen,
%welche im Rahmen des Drittsemestermoduls 0.3.2 RESO: Resources:
%Financial Resources, Human Resources, Organisation wiederkehren
%werden.
%}}}

\vspace{10mm}
\noindent
In this chapter we want to provide a brief introduction into some 
basic concepts of {\bf financial mathematics}. As we will try to 
emphasise, many applications of these concepts (that have 
immediate practical relevance) are founded on only two simple and 
easily accessible mathematical structures: the so-called 
arithmetical and geometrical real-valued sequences and their 
associated finite series.

%%%%%%%%%%%%%%%%%%%%%%%%%%%%%%%%%%%%%%%%%%%%%%%%%%%%%%%%%%%%%%%%%%%
\section[Arithmetical and geometrical sequences and series]%
{Arithmetical and geometrical sequences and series}
\lb{sec:folgreih}
%%%%%%%%%%%%%%%%%%%%%%%%%%%%%%%%%%%%%%%%%%%%%%%%%%%%%%%%%%%%%%%%%%%
%------------------------------------------------------------------
\subsection{Arithmetical sequence and series}
\lb{subsec:arithseq}
%------------------------------------------------------------------
An {\bf arithmetical sequence} of $n \in \mathbb{N}$ real 
numbers~$a_{n} \in \mathbb{R}$,
%
\[
(a_{n})_{n \in \mathbb{N}} \ ,
\]
%
is defined by the property that the {\bf difference} $d$ between 
neighbouring elements in the sequence be \emph{constant}, i.e., 
for  $n>1$
%
\be
\lb{arifolrek}
\fbox{$\displaystyle
a_{n}-a_{n-1}=:d=\text{constant}\neq 0 \ ,
$}
\ee
%
with $a_{n}, a_{n-1}, d \in \mathbb{R}$. Given this recursive 
formation rule, one may infer the {\bf explicit representation} of 
an {\bf arithmetical sequence} as
%
\be
\lb{arifolex}
a_{n} = a_{1} + (n-1)d
\quad\text{with}\quad
n \in \mathbb{N} \ .
\ee
%
Note that any {\bf arithmetical sequence} is \emph{uniquely 
determined} by the two free parameters $a_{1}$ and $d$, the 
starting value of the sequence and the constant difference between 
neighbours in the sequence, respectively. 
Equation~(\ref{arifolex}) shows that the elements $a_{n}$ in a 
non-trivial {\bf arithmetical sequence} exhibit either {\bf 
linear} growth or {\bf linear} decay with $n$.

\medskip
\noindent
When one calculates for an {\bf arithmetical sequence} of $n+1$ 
real numbers the {\bf arithmetical mean} of the immediate 
neighbours of any particular element $a_{n}$ (with $n \geq 2$), 
one finds that
%
\be
\frac{1}{2}\,(a_{n-1}+a_{n+1})
= \frac{1}{2}\left(a_{1}+(n-2)d+a_{1}+nd\right)
= a_{1}+(n-1)d
= a_{n} \ .
\ee
%

\medskip
\noindent
Summation of the first $n$ elements of an arbitrary {\bf 
arithmetical sequence} of real numbers leads to a {\bf finite 
arithmetical series},
%
\be
S_{n} := a_{1}+a_{2}+\ldots+a_{n} = \sum_{k=1}^{n}a_{k}
= \sum_{k=1}^{n}\left[a_{1} + (k-1)d\right]
= na_{1}+ \frac{d}{2}\,(n-1)n \ .
\ee
%
In the last algebraic step use was made of the Gau\ss ian {\bf 
identity}\footnote{Analogously, the modified Gau\ss ian identity 
$\displaystyle \sum_{k=1}^{n}(2k-1) \equiv n^{2}$ applies.} (cf., 
e.g., Bosch (2003)~\ct[p~21]{bos2003})
%
\be
\lb{id1}
\fbox{$\displaystyle
\sum_{k=1}^{n-1}k \equiv \frac{1}{2}\,(n-1)n \ .
$}
\ee
%

%------------------------------------------------------------------
\subsection{Geometrical sequence and series}
\lb{subsec:geomseq}
%------------------------------------------------------------------
A {\bf geometrical sequence} of $n \in \mathbb{N}$ real 
numbers~$a_{n} \in \mathbb{R}$,
%
\[
(a_{n})_{n \in \mathbb{N}} \ ,
\]
%
is defined by the property that the {\bf quotient} $q$ between 
neighbouring elements in the sequence be \emph{constant}, i.e., 
for $n>1$
%
\be
\lb{geofolrek}
\fbox{$\displaystyle
\frac{a_{n}}{a_{n-1}}=:q=\text{constant}\neq 0 \ ,
$}
\ee
%
with $a_{n}, a_{n-1} \in \mathbb{R}$ and $q \in 
\mathbb{R}\backslash\{0,1\}$. Given this recursive formation rule, 
one may infer the {\bf explicit representation} of a {\bf 
geometrical sequence} as
%
\be
\lb{geofolex}
a_{n} = a_{1}q^{n-1}
\quad\text{with}\quad
n \in \mathbb{N}.
\ee
%
Note that any {\bf geometrical sequence} is \emph{uniquely 
determined} by the two free parameters $a_{1}$ and $q$, the 
starting value of the sequence and the constant quotient between 
neighbours in the sequence, respectively. 
Equation~(\ref{geofolex}) shows that the elements $a_{n}$ in a 
non-trivial {\bf geometrical sequence} exhibit either {\bf 
exponential} growth or {\bf exponential} decay with $n$ (cf. 
Sec.~\ref{subsec:exponentials}).

\medskip
\noindent
When one calculates for a {\bf geometrical sequence} of $n+1$ real 
numbers the {\bf geometrical mean} of the immediate neighbours of 
any particular element $a_{n}$ (with $n \geq 2$), one finds that
%
\be
\sqrt{a_{n-1}\cdot a_{n+1}}
= \sqrt{a_{1}q^{n-2}\cdot a_{1}q^{n}}
= a_{1}q^{n-1}
= a_{n} \ .
\ee
%

\medskip
\noindent
Summation of the first $n$ elements of an arbitrary {\bf 
geometrical sequence} of real numbers leads to a {\bf finite 
geometrical series},
%
\be
S_{n} := a_{1}+a_{2}+\ldots+a_{n} = \sum_{k=1}^{n}a_{k}
= \sum_{k=1}^{n}\left[a_{1}q^{k-1}\right]
= a_{1}\sum_{k=0}^{n-1}q^{k}
= a_{1}\,\frac{q^{n}-1}{q-1} \ .
\ee
%
In the last algebraic step use was made of the {\bf identity} (cf., e.g., Bosch (2003)~\ct[p~27]{bos2003})
%
\be
\lb{id2}
\fbox{$\displaystyle
\sum_{k=0}^{n-1}q^{k} \equiv \frac{q^{n}-1}{q-1}
\quad\text{for}\quad
q \in \mathbb{R}\backslash\{0,1\} \ .
$}
\ee
%

%%%%%%%%%%%%%%%%%%%%%%%%%%%%%%%%%%%%%%%%%%%%%%%%%%%%%%%%%%%%%%%%%%%
\section[Interest and compound interest]%
{Interest and compound interest}
\lb{sec:zins}
%%%%%%%%%%%%%%%%%%%%%%%%%%%%%%%%%%%%%%%%%%%%%%%%%%%%%%%%%%%%%%%%%%%
Let us consider a first rather simple interest model. Suppose 
given an {\bf initial capital} of positive value $K_{0} > 0~{\rm 
CU}$ paid into a bank account at some initial instant, and a time 
interval consisting of $n \in \mathbb{N}$ {\bf periods} of equal 
lengths. At the end of each period, the money in this bank account 
shall earn a service fee corresponding to an {\bf interest rate} 
of $p>0$ percent. Introducing the dimensionless {\bf interest 
factor}\footnote{Inverting this defining relation for $q$ leads to 
$p=100\cdot(q-1)$.}
%
\be
q:=1+\frac{p}{100} > 1 \ ,
\ee
%
one finds that by the end of the first interest period a total 
capital of value (in CU)
%
\[
K_{1} = K_{0} + K_{0}\cdot\frac{p}{100}
= K_{0}\left(1+\frac{p}{100}\right) = K_{0}q
\]
%
will have accumulated. When the entire time interval of $n$ 
interest periods has ended, a {\bf final capital} worth of (in CU)
%
\be
\lb{kap1}
\fbox{$\displaystyle
\text{recursively:}\ K_{n}=K_{n-1}q \ , \quad
n\in\mathbb{N} \ ,
$}
\ee
%
will have accumulated, where $K_{n-1}$ denotes the capital (in CU) 
accumulated by the end of $n-1$ interest periods. This recursive 
representation of the growth of the initial capital $K_{0}$ due to 
a total of $n$ interest payments and the effect of {\bf compound 
interest} makes explicit the direct link with the mathematical 
structure of a {\bf geometrical sequence} of real 
numbers~(\ref{geofolrek}).

\medskip
\noindent
It is a straightforward exercise to show that in this simple 
interest model the final capital $K_{n}$ is related to the initial 
capital $K_{0}$ by
%
\be
\lb{kap2}
\fbox{$\displaystyle
\text{explicitly:}\ K_{n}=K_{0}q^{n} \ , \quad
n\in\mathbb{N} \ .
$}
\ee
%
Note that this equation links the four non-negative quantities 
$K_{n}$, $K_{0}$, $q$ and $n$ to one another. Hence, knowing the 
values of three of these quantities, one may solve 
Eq.~(\ref{kap2}) to obtain the value of the fourth. For example, 
solving Eq.~(\ref{kap2}) for $K_{0}$ yields
%
\be
K_{0} = \frac{K_{n}}{q^{n}} =: B_{0} \ .
\ee
%
In this particular variant, $K_{0}$ is referred to as the {\bf 
present value} $B_{0}$ of the final capital $K_{n}$; this is 
obtained from $K_{n}$ by an $n$-fold division with the interest 
factor $q$. 

\medskip
\noindent
Further possibilities of re-arranging Eq.~(\ref{kap2}) are:
%
\begin{itemize}
\item[(i)] Solving for the {\bf interest factor} $q$:
%
\be
q = \sqrt[n]{\frac{K_{n}}{K_{0}}} \ ,
\ee
%
\item[(ii)] Solving for the {\bf contract period} $n$:
%
\be
n = \frac{\ln\left(K_{n}/K_{0}\right)}{\ln(q)} \ .
\ee
%
\end{itemize}
%
{}From now on, $n \in \mathbb{N}$ shall denote the number of full years that have passed in a specific interest model.

\medskip
\noindent
Now we turn to discuss a second, more refined interest model. Let 
us suppose that an {\bf initial capital} $K_{0}>0~{\rm CU}$ earns 
interest during one full year $m \in \mathbb{N}$ times at the 
$m$th part of a {\bf nominal annual interest rate} $p_{\rm 
nom}>0$. At the end of the first out of $m$ periods of equal 
length $1/m$, the initial capital $K_{0}$ will thus have increased 
to an amount
%
\[
K_{1/m} = K_{0} + K_{0}\cdot\frac{p_{\rm nom}}{m\cdot 100}
= K_{0}\left(1+\frac{p_{\rm nom}}{m\cdot 100}\right) \ .
\]
%
By the end of the $k$th ($k \leq m$) out of $m$ periods the {\bf 
account balance} will have become
%
\[
K_{k/m} = K_{0}\left(1+\frac{p_{\rm nom}}{m\cdot 100}\right)^{k}
\ ;
\]
%
the interest factor $\displaystyle
\left(1+\frac{p_{\rm nom}}{m\cdot 100}\right)$ will then have been 
applied $k$ times to $K_{0}$. At the end of the full year, $K_{0}$ 
in this interest model will have increased to
%
\[
K_{1} = K_{m/m} = K_{0}\left(1+\frac{p_{\rm nom}}{m\cdot 100}
\right)^{m} \ , \quad
m \in \mathbb{N} \ .
\]
%
This relation defines an {\bf effective interest factor}
%
\be
\lb{qeff}
q_{\rm eff} := \left(1+\frac{p_{\rm nom}}{m\cdot 100}
\right)^{m} \ ,
\ee
%
with associated {\bf effective annual interest rate}
%
\be
\lb{peff}
\fbox{$\displaystyle
p_{\rm eff} = 100\cdot\left[\left(1+\frac{p_{\rm nom}}{m\cdot 100}
\right)^{m}-1\right] \ , \quad m\in\mathbb{N} \ ,
$
}
\ee
%
obtained from re-arranging $\displaystyle q_{\rm 
eff}=1+\frac{p_{\rm eff}}{100}$.

\medskip
\noindent
When, ultimately, $n \in \mathbb{N}$ full years will have passed 
in the second interest model, the initial capital~$K_{0}$ will 
have been transformed into a final capital of value
%
\be
\lb{kap3}
\fbox{$\displaystyle
K_{n}=K_{0}\left(1+\frac{p_{\rm nom}}{m\cdot 100}\right)^{n\cdot m}
= K_{0}q_{\rm eff}^{n} \ , \quad n,m\in\mathbb{N} \ .
$}
\ee
%
The {\bf present value} $B_{0}$ of $K_{n}$ is thus given by
%
\be
B_{0} = \frac{K_{n}}{q_{\rm eff}^{n}} = K_{0} \ .
\ee
%

\medskip
\noindent
Finally, as a third interest model relevant to applications in 
{\bf Finance}, we turn to consider the concept of {\bf installment 
savings}. For simplicity, let us restrict our discussion to the 
case when $n \in \mathbb{N}$ equal {\bf installments} of 
\emph{constant} value $E>0~{\rm CU}$ are paid into an account that 
earns $p>0$ percent annual interest (i.e., $q>1$) at the beginning 
of each of $n$~full years. The initial account balance be 
$K_{0}=0~{\rm CU}$. At the end of a first full year in this 
interest model, the account balance will have increased to
%
\[
K_{1} = E + E\cdot\frac{p}{100}
= E\left(1+\frac{p}{100}\right)
= Eq \ .
\]
%
At the end of two full years one finds, substituting for $K_{1}$,
%
\[
K_{2} = (K_{1}+E)q = (Eq+E)q = E(q^{2}+q) = Eq(q+1) \ .
\]
%
At the end of $n$ full years we have, recursively substituting for $K_{n-1}$, $K_{n-2}$, etc.,
%
\[
K_{n} = (K_{n-1}+E)q
= \cdots
=E(q^{n}+\ldots+q^{2}+q)
=Eq(q^{n-1}+\ldots+q+1)
= Eq\sum_{k=0}^{n-1}q^{k} \ .
\]
%
Using the identity~(\ref{id2}), since presently $q>1$, the {\bf 
account balance} at the end of $n$ full years can be reduced to 
the expression
%
\be
\lb{kap4}
\fbox{$\displaystyle
K_{n}=Eq\,\frac{q^{n}-1}{q-1} \ , \quad
q \in \mathbb{R}_{>1} \ , \quad
n\in\mathbb{N} \ .
$}
\ee
%
The {\bf present value} $B_{0}$ associated with $K_{n}$ is 
obtained by $n$-fold division of $K_{n}$ with the interest 
factor~$q$:
%
\be
B_{0} := \frac{K_{n}}{q^{n}}
\overbrace{=}^{\text{Eq.}~\ref{kap4}}
 = \frac{E(q^{n}-1)}{q^{n-1}(q-1)} \ .
\ee
%
This gives the value of an initial capital $B_{0}$ which will grow 
to the \emph{same} final value $K_{n}$ after $n$ annual interest 
periods with constant interest factor $q>1$.

\medskip
\noindent
Lastly, re-arranging Eq.~(\ref{kap4}) to solve for the {\bf 
contract period} $n$ yields.
%
\be
n = \frac{\ln\left[1+(q-1)(K_{n}/Eq)\right]}{\ln(q)} \ .
\ee
%

%%%%%%%%%%%%%%%%%%%%%%%%%%%%%%%%%%%%%%%%%%%%%%%%%%%%%%%%%%%%%%%%%%%
\section[Redemption payments in constant annuities]%
{Redemption payments in constant annuities}
\lb{sec:tilg}
%%%%%%%%%%%%%%%%%%%%%%%%%%%%%%%%%%%%%%%%%%%%%%%%%%%%%%%%%%%%%%%%%%%
The starting point of the next discussion be a {\bf mortgage loan} 
of amount $R_{0}>0~{\rm CU}$ that an {\bf economic agent} borrowed 
from a bank at the obligation of annual service payments of $p>0$ 
percent (i.e., $q>1$) on the {\bf remaining debt}. We suppose that 
the contract between the agent and the bank fixes the following 
conditions:
%
\begin{itemize}
\item[(i)]~the first {\bf redemption payment} $T_{1}$ amount to 
$t>0$ percent of the mortgage $R_{0}$,

\item[(ii)]~the remaining debt shall be paid back to the bank in 
\emph{constant} {\bf annuities} of value $A>0~{\rm CU}$ at the end 
of each full year that has passed.
\end{itemize}
%
The {\bf annuity} $A$ is defined as the \emph{sum} of the variable 
$n$th {\bf interest payment} $Z_{n} > 0~{\rm CU}$ and the variable 
$n$th {\bf redemption payment} $T_{n} > 0~{\rm CU}$. In the 
present model we impose on the annuity the condition that it be 
\emph{constant} across full years,
%
\be
\lb{annuity1}
A = Z_{n} + T_{n}
\stackrel{!}{=} \text{constant} \ .
\ee
%
For $n=1$, for example, we thus obtain
%
\be
\lb{annuity2}
A = Z_{1} + T_{1}
= R_{0}\cdot\frac{p}{100} + R_{0}\cdot\frac{t}{100}
= R_{0}\left(\frac{p+t}{100}\right)
= R_{0}\left[(q-1)+\frac{t}{100}\right]
\stackrel{!}{=} \text{constant} \ .
\ee
%
For the first full year of a running mortgage contract, the 
interest payment, the redemption payment, and, following the 
payment of a first annuity, the remaining debt take the values
%
\begin{eqnarray*}
Z_{1} & = & R_{0}\cdot\frac{p}{100} \ = \ R_{0}(q-1) \\
%
T_{1} & = & A - Z_{1} \\
%
R_{1} & = & R_{0} + Z_{1} - A
\ \overbrace{=}^{\text{substitute for}\ Z_{1}}
\ R_{0} + R_{0}\cdot\frac{p}{100} - A
\ = \ R_{0}q - A \ .
\end{eqnarray*}
%
By the end of a second full year, these become
%
\begin{eqnarray*}
Z_{2} & = & R_{1}(q-1) \\
%
T_{2} & = & A - Z_{2} \\
%
R_{2} & = & R_{1} + Z_{2} - A
\ \overbrace{=}^{\text{substitute for}\ Z_{2}}
\ R_{1}q - A
\ \overbrace{=}^{\text{substitute for}\ R_{1}}
\ R_{0}q^{2} - A(q+1) \ .
\end{eqnarray*}
%
At this stage, it has become clear according to which patterns the 
different quantities involved in the redemption payment model need 
to be formed. The {\bf interest payment} for the $n$th full year 
in a mortgage contract of constant anuities amounts to 
(recursively)
%
\be
\lb{zinsn}
Z_{n}=R_{n-1}(q-1) \ , \quad n\in\mathbb{N} \ ,
\ee
%
where $R_{n-1}$ denotes the remaining debt at the end of the 
previous full year. The {\bf redemption payment} for full year $n$ 
is then given by (recursively)
%
\be
\lb{tilgn}
T_{n} = A - Z_{n} \ , \quad n\in\mathbb{N} \ .
\ee
%
The {\bf remaining debt} at the end of the $n$th full year then is 
(in CU)
%
\be
\lb{remdebtrek}
\fbox{$\displaystyle
\text{recursively:} \quad
R_{n} = R_{n-1}+Z_{n}-A
\ = \ R_{n-1}q-A \ , \quad n\in\mathbb{N} \ .
$}
\ee
%
By successive backward substitution for $R_{n-1}$, $R_{n-2}$, etc.,
$R_{n}$ can be re-expressed as
%
\[
R_{n} = R_{0}q^{n}-A(q^{n-1}+\ldots+q+1)
= R_{0}q^{n}-A\sum_{k=0}^{n-1}q^{k} \ .
\]
%
Now employing the identity~(\ref{id2}), we finally obtain (since $q>1$)
%
\be
\lb{remdebtex}
\fbox{$\displaystyle
\text{explicitly:} \quad
R_{n} = R_{0}q^{n}-A\,\frac{q^{n}-1}{q-1} \ , \quad
n\in\mathbb{N} \ .
$}
\ee
%

\medskip
\noindent
All the formulae we have now derived
%, in the present context,
for computing the values of the quantities $\{n, Z_{n}, T_{n}, 
R_{n}\}$ form the basis of a formal {\bf redemption payment plan}, 
given by
%
\begin{center}
		\begin{tabular}{c||c|c|c}
		$n$ & $Z_{n}$ [CU] & $T_{n}$ [CU] & $R_{n}$ [CU] \\
		\hline\hline
		$0$ & -- & -- & $R_{0}$ \\
		$1$ & $Z_{1}$ & $T_{1}$ & $R_{1}$ \\
		$2$ & $Z_{2}$ & $T_{2}$ & $R_{2}$ \\
		\vdots & \vdots & \vdots & \vdots
		\end{tabular} \ ,
\end{center}
%
a standard scheme that banks must make available to their mortgage 
customers for the purpose of financial orientation.

\medskip
\noindent
\underline{\bf Remark:} For known values of the free parameters 
$R_{0}>0~{\rm CU}$, $q>1$ and $A>0~{\rm CU}$, the simple recursive 
formulae (\ref{zinsn}), (\ref{tilgn}) and (\ref{remdebtrek}) can 
be used to implement a redemption payment plan in a modern 
spreadsheet programme such as EXCEL or OpenOffice.

\medskip
\noindent
We emphasise the following observation concerning 
Eq.~(\ref{remdebtex}): since the constant annuity $A$ contains 
implicitly a factor $(q-1)$ [cf. Eq.~(\ref{annuity2})], the two 
competing terms in this relation each grow exponentially with $n$. 
For the redemption payments to eventually terminate, it is thus 
essential to fix the free parameter $t$ (for known $p>0 
\Leftrightarrow q>1$) in such a way that the second term on the 
right-hand side of Eq.~(\ref{remdebtex}) is given the possibility 
to catch up with the first as $n$ progresses (the latter of which 
has a head start of $R_{0}>0~{\rm CU}$ at $n=0$). The necessary 
condition following from the requirement that $R_{n} 
\stackrel{!}{\leq} R_{n-1}$ is thus $t>0$.

\medskip
\noindent
Equation~(\ref{remdebtex}) links the five non-negative quantities 
$R_{n}$, $R_{0}$, $q$, $n$ and $A$ to one another. Given one knows 
the values of four of these, one can solve for the fifth. For 
example:
%
\begin{itemize}
\item[(i)] Calculation of the {\bf contract period} $n$ of a 
mortgage contract, knowing the mortgage $R_{0}$, the interest 
factor $q$ and the annuity $A$. Solving the condition  $R_{n} 
\stackrel{!}{=} 0$ imposed on $R_{n}$ for $n$ yields (after a few 
algebraic steps)
%
\be
n = \frac{\ln\left(1+\frac{p}{t}\right)}{\ln(q)} \ ;
\ee
%
the contract period is thus independent of the value of the 
mortgage loan, $R_{0}$.

\item[(ii)] Evaluation of the {\bf annuity} $A$, knowing the 
contract period $n$, the mortgage loan~$R_{0}$, and the interest 
factor $q$. Solving the condition $R_{n} \stackrel{!}{=} 0$ 
imposed on $R_{n}$ for $A$ immediately yields
%
\be
\lb{annuity3}
A = \frac{q^{n}(q-1)}{q^{n}-1}\,R_{0} \ .
\ee
%
Now equating the two expressions (\ref{annuity3}) and
(\ref{annuity2}) for the annuity $A$, one finds in addition that
%
\be
\frac{t}{100} = \frac{q-1}{q^{n}-1} \ .
\ee
%
\end{itemize}
%

%%%%%%%%%%%%%%%%%%%%%%%%%%%%%%%%%%%%%%%%%%%%%%%%%%%%%%%%%%%%%%%%%%%
\section[Pension calculations]{Pension calculations}
\lb{sec:rente}
%%%%%%%%%%%%%%%%%%%%%%%%%%%%%%%%%%%%%%%%%%%%%%%%%%%%%%%%%%%%%%%%%%%
Quantitative models for {\bf pension calculations} assume given an 
{\bf initial capital} $K_{0}>0~{\rm CU}$ that was paid into a bank 
account at a particular moment in time. The issue is to monitor 
the subsequent evolution in {\bf discrete time} $n$ of the {\bf 
account balance} $K_{n}$ (in CU), which is subjected to two 
competing influences: on the one-hand side, the bank account earns 
interest at an {\bf annual interest rate} of $p>0$ percent (i.e., 
$q>1$), on the other, it is supposed that throughout one full year 
a total of~$m \in \mathbb{N}$ pension payments of the 
\emph{constant} {\bf amount}~$a$ are made from this bank account, 
always at the beginning of each of $m$ intervals of equal duration 
per year.

\medskip
\noindent
Let us begin by evaluating the amount of interest earned per year 
by the bank account. An important point in this respect is the 
fact that throughout one full year there is a total of $m$ 
deductions of value $a$ from the bank account, i.e., in general 
the account balance does \emph{not} stay constant throughout that 
year but rather decreases in discrete steps. For this reason, the 
account is credited by the bank with interest only at the $m$th 
part of $p>0$ percent for each interval (out of the total of $m$) 
that has passed, with \emph{no} compound interest effect. Hence, 
at the end of the first out of $m$ intervals per year the bank 
account has earned interest worth of (in CU)
%
\[
Z_{1/m} = (K_{0}-a)\cdot\frac{p}{m\cdot 100}
= (K_{0}-a)\,\frac{(q-1)}{m} \ .
\]
%
The interest earned for the $k$th interval (out of $m$; $k \leq m$) is then given by
%
\[
Z_{k/m} = (K_{0}-ka)\,\frac{(q-1)}{m} \ .
\]
%
Summation over the contributions of each of the $m$ intervals to 
the interest earned then yields for the entire interest earned 
during the first full year (in CU)
%
\[
Z_{1} = \sum_{k=1}^{m}Z_{k/m}
= \sum_{k=1}^{m}(K_{0}-ka)\,\frac{(q-1)}{m}
= \frac{(q-1)}{m}\left[mK_{0}-a\sum_{k=1}^{m}k\right] \ .
\]
%
By means of substitution from the identity~(\ref{id1}), this 
result can be recast into the equivalent form
%
\be
\lb{rentezins1}
Z_{1} = \left[K_{0}-\frac{1}{2}\,(m+1)a\right](q-1) \ .
\ee
%
Note that this quantity decreases linearly with the number of 
deductions $m$ made per year resp.~with the pension payment 
amount~$a$.

\medskip
\noindent
One now finds that the account balance at the end of the first 
full year that has passed is given by
%
\[
K_{1} = K_{0} - ma + Z_{1}
\overbrace{=}^{\text{Eq.}~(\ref{rentezins1})} K_{0}q
- \left[m+\frac{1}{2}\,(m+1)(q-1)\right]a \ .
\]
%
At the end of a second full year of the pension payment contract the interest earned is
%
\[
Z_{2} = \left[K_{1}-\frac{1}{2}\,(m+1)a\right](q-1) \ ,
\]
%
while the account balance amounts to
%
\[
K_{2} = K_{1} - ma + Z_{2}
\overbrace{=}^{\text{substitute for}\ K_{1}
\ \text{and}\ Z_{2}} K_{0}q^{2}
- \left[m+\frac{1}{2}\,(m+1)(q-1)\right]a(q+1) \ .
\]
%
At this stage, certain fairly simple patterns for the {\bf 
interest earned} during full year $n$, and the {\bf account 
balance} after $n$ full years, reveal themselves. For $Z_{n}$ we 
have
%
\be
Z_{n} = \left[K_{n-1}-\frac{1}{2}\,(m+1)a\right](q-1) \ ,
\ee
%
and for $K_{n}$ one obtains

%
\[
K_{n} = K_{n-1} - ma + Z_{n}
\overbrace{=}^{\text{substitute for}\ K_{n-1}
\ \text{and}\ Z_{n}} K_{0}q^{n}
- \left[m+\frac{1}{2}\,(m+1)(q-1)\right]a\sum_{k=0}^{n-1}q^{k} \ .
\]
%
The latter result can be re-expressed upon substitution from the 
identity~(\ref{id2}). Thus, $K_{n}$ can finally be given by
%
\be
\lb{pensionex}
\fbox{$\displaystyle
\text{explicitly:} \quad
K_{n} = K_{0}q^{n} - \left[m+\frac{1}{2}\,(m+1)(q-1)\right]
a\,\frac{q^{n}-1}{q-1} \ , \quad
n, m \in \mathbb{N} \ .
$
}
\ee
%
In a fashion practically identical to our discussion of the 
redemption payment model in Sec.~\ref{sec:tilg}, the two competing 
terms on the right-hand side of Eq.~(\ref{pensionex}) likewise 
exhibit exponential growth with the number $n$ of full years 
passed. Specifically, it depends on the values of the parameters 
$K_{0} > 0~{\rm CU}$, $q>1$, $a>0~{\rm CU}$, as well as $m \geq 1$,
whether the second term eventually manages to catch up with the 
first as $n$ progresses (the latter of which, in this model, is 
given a head start of value $K_{0} > 0~{\rm CU}$ at $n=0$).

\medskip
\noindent
We remark that Eq.~(\ref{pensionex}), again, may be algebraically 
re-arranged at one's convenience (as long as division by zero is 
avoided). For example:
%
\begin{itemize}

\item[(i)] The {\bf duration} $n$ (in full years) of a particular 
pension contract is obtained from solving the condition 
$K_{n}\stackrel{!}{=}0$ accordingly. Given that 
$[\ldots]a-K_{0}(q-1)>0$, one thus finds\footnote{To avoid 
notational overload, the brackets $[\ldots]$ here represent the 
term $\left[m+\frac{1}{2}\,(m+1)(q-1)\right]$.}
%
\be
\lb{rentelaufzeit}
n = \frac{\ln\left(\frac{[\ldots]a}{
[\ldots]a-K_{0}(q-1)}\right)}{\ln(q)} \ .
\ee
%

\item[(ii)] The {\bf present value} $B_{0}$ of a pension scheme 
results from the following consideration: for fixed interest 
factor $q>1$, which initial capital $K_{0} > 0~{\rm CU}$ must be 
paid into a bank account such that for a duration of $n$ full 
years one can receive payments of constant amount $a$ at the 
beginning of each of $m$ intervals (of equal length) per year? The 
value of $B_{0}=K_{0}$ is again obtained from imposing on 
Eq.~(\ref{pensionex}) the condition $K_{n}\stackrel{!}{=}0$ and 
solving for $K_{0}$. This yields
%
\be
\lb{rentebarwert}
B_{0} = K_{0} = \left[m+\frac{1}{2}\,(m+1)(q-1)\right]a\,
\frac{q^{n}-1}{q^{n}(q-1)} \ .
\ee
%

\item[(iii)] The idea of so-called {\bf everlasting pension 
payments} of amount $a_{\rm ever} >0~{\rm CU}$ is based on the 
strategy to consume only the annual interest earned by an initial 
capital $K_{0}>0~{\rm CU}$ residing in a bank account with 
interest factor $q>1$. Imposing now on Eq.~(\ref{pensionex}) the 
condition $K_{n}\stackrel{!}{=}K_{0}$ to hold for all values of 
$n$, and then solving for $a$, yields the result
%
\be
a_{\rm ever} = \frac{q-1}{m+\frac{1}{2}\,(m+1)(q-1)}\,K_{0} \ ;
\ee
%
Note that, naturally, $a_{\rm ever}$ is directly proportional to 
the initial capital $K_{0}$!

\end{itemize}
%

%%%%%%%%%%%%%%%%%%%%%%%%%%%%%%%%%%%%%%%%%%%%%%%%%%%%%%%%%%%%%%%%%%%
\section[Linear and declining-balance depreciation methods]%
{Linear and declining-balance depreciation methods}
\lb{sec:abschr}
%%%%%%%%%%%%%%%%%%%%%%%%%%%%%%%%%%%%%%%%%%%%%%%%%%%%%%%%%%%%%%%%%%%
Attempts at the quantitative description of the process of 
declining material value of industrial goods, properties or other 
assets, of {\bf initial value} $K_{0}>0~{\rm CU}$, are referred to 
as {\bf depreciation}. International tax laws generally provide 
investors with a choice between two particular mathematical 
methods of calculating {\bf depreciation}. We will discuss 
these options in turn.

%------------------------------------------------------------------
\subsection{Linear depreciation method}
%------------------------------------------------------------------
When the {\bf initial value} $K_{0}>0~{\rm CU}$ is supposed to 
decline to $0~{\rm CU}$ in the space of $N$ full years by equal 
annual amounts, the {\bf remaining value} $R_{n}$ (in CU) at the 
end of $n$ full years is described by
%
\be
\fbox{$\displaystyle
R_{n} = K_{0} - n\left(\frac{K_{0}}{N}\right) \ ,
\quad n=1,\ldots,N \ .
$}
\ee
%
Note that for the difference of remaining values for years 
adjacent one obtains $\displaystyle R_{n}-R_{n-1}
= -\left(\frac{K_{0}}{N}\right) =: d < 0$. The underlying 
mathematical structure of the {\bf straight line depreciation 
method} is thus an {\bf arithmetical sequence} of real numbers, 
with constant \emph{negative} difference $d$ between neighbouring 
elements (cf. Sec.~\ref{subsec:arithseq}).

%------------------------------------------------------------------
\subsection{Declining-balance depreciation method}
%------------------------------------------------------------------
The foundation of the second depreciation method to be described 
here, for an industrial good of {\bf initial value} $K_{0}>0~{\rm 
CU}$, is the idea that per year the value declines by a certain 
{\bf percentage rate} $p>0$ of the value of the good during the 
previous year. Introducing a dimensionless {\bf depreciation 
factor} by
%
\be
q := 1-\frac{p}{100} < 1 \ ,
\ee
%
the {\bf remaining value} $R_{n}$ (in CU) after $n$ full years 
amounts to
%
\be
\lb{deprrec}
\fbox{$\displaystyle
\text{recursively:} \quad
R_{n} = R_{n-1}q \ , \quad R_{0} \equiv K_{0} \ ,\quad
n\in\mathbb{N} \ .
$}
\ee
%
The underlying mathematical structure of the {\bf declining 
balance depreciation method} is thus a {\bf geo\-metrical 
sequence} of real numbers, with constant ratio $0<q<1$ between 
neighbouring elements (cf. Sec.~\ref{subsec:geomseq}). With 
increasing $n$ the values of these elements become ever smaller. 
By means of successive backward substitution 
expression~(\ref{deprrec}) can be transformed to
%
\be
\lb{deprex}
\fbox{$\displaystyle
\text{explicitly:} \quad
R_{n} = K_{0}q^{n} \ , \quad 0 < q < 1 \ , \quad
n\in\mathbb{N} \ .
$}
\ee
%

\medskip
\noindent
{}From Eq.~(\ref{deprex}), one may derive results concerning the 
following questions of a quantitative nature:

%
\begin{itemize}

\item[(i)] Suppose given a depreciation factor $q$ and a projected 
remaining value $R_{n}$ for some industrial good. After which {\bf 
depreciation period} $n$ will this value be attained? One finds
%
\be
n = \frac{\ln\left(R_{n}/K_{0}\right)}{\ln(q)} \ .
\ee
%

\item[(ii)] Knowing a projected depreciation period $n$ and 
corresponding remaining value $R_{n}$, at which {\bf percentage 
rate} $p>0$ must the depreciation method be operated? This yields
%
\be
q = \sqrt[n]{\frac{R_{n}}{K_{0}}}
\quad\Rightarrow\quad
p = 100 \cdot \left(1-\sqrt[n]{\frac{R_{n}}{K_{0}}}\right) \ .
\ee
%

\end{itemize}
%

%%%%%%%%%%%%%%%%%%%%%%%%%%%%%%%%%%%%%%%%%%%%%%%%%%%%%%%%%%%%%%%%%%%
\section[Summarising formula]{Summarising formula}
\lb{sec:zus}
%%%%%%%%%%%%%%%%%%%%%%%%%%%%%%%%%%%%%%%%%%%%%%%%%%%%%%%%%%%%%%%%%%%
To conclude this chapter, let us summarise the results on {\bf 
elementary financial mathematics} that we derived along the way. 
Remarkably, these can be condensed in a single formula which 
contains the different concepts discussed as special cases. This 
formula, in which $n$ represents the number of full years that 
have passed, is given by (cf. Zeh--Marschke (2010) \ct{zeh2010}):
%
\be
\lb{fincmastereq}
\fbox{$\displaystyle
K_{n} = K_{0}q^{n} + R\,\frac{q^{n}-1}{q-1} \ , \quad
q \in \mathbb{R}_{>0}\backslash\{1\} \ , \quad
n\in\mathbb{N} \ .
$}
\ee
%

\medskip
\noindent
The different special cases contained therein are:
%
\begin{itemize}
\item[(i)] {\bf Compound interest} for an initial capital 
$K_{0}>0~{\rm CU}$: with $R=0$ and $q>1$, Eq.~(\ref{fincmastereq}) 
reduces to Eq.~(\ref{kap2}).

\item[(ii)] {\bf Installment savings} with constant installments 
$E>0~{\rm CU}$: with $K_{0}=0~{\rm CU}$, $q>1$ and $R=Eq$,
Eq.~(\ref{fincmastereq}) reduces to Eq.~(\ref{kap4}).

\item[(iii)] {\bf Redemption payments in constant annuities}:
with $K_{0}=-R_{0}<0~{\rm CU}$, $q>1$ and $R=A>0~{\rm CU}$, 
Eq.~(\ref{fincmastereq}) reduces to the \emph{negative (!)} of 
Eq.~(\ref{remdebtex}). In this dual formulation, remaining debt 
$K_{n}=-R_{n}$ is (meaningfully) expressed as a negative account 
balance.

\item[(iv)] {\bf Pension payments} of constant amount $a>0~{\rm 
CU}$: with $q>1$ and $\displaystyle R=-\left[m
+\frac{1}{2}\,(m+1)(q-1)\right]a$, Eq.~(\ref{fincmastereq}) 
transforms to Eq.~(\ref{pensionex}).

\item[(v)] {\bf Declining balance depreciation} of an asset of 
initial value $K_{0}>0~{\rm CU}$: with $R=0$ and $0<q<1$, 
Eq.~(\ref{fincmastereq}) converts to Eq.~(\ref{deprex}) for the 
remaining value $K_{n}=R_{n}$.

\end{itemize}
%

%%%%%%%%%%%%%%%%%%%%%%%%%%%%%%%%%%%%%%%%%%%%%%%%%%%%%%%%%%%%%%%%%%%
%%%%%%%%%%%%%%%%%%%%%%%%%%%%%%%%%%%%%%%%%%%%%%%%%%%%%%%%%%%%%%%%%%%
